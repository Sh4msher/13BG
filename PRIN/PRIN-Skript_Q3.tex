% ================= Terminal-Style Template mit Box-zu-Box-Farbverlauf =================
\documentclass[11pt,a4paper,oneside]{article}

% ---------- Fonts & Math (Xe/Lua) ----------
\usepackage{fontspec}
\usepackage{unicode-math}

% --- Hauptschrift ---
\setmainfont{IBM Plex Sans}
%\setsansfont{IBM Plex Sans}

% --- Mathematik ---
\setmathfont{Libertinus Math}

% --- Monospace / Terminal ---
%\setmonofont{Source Code Pro}[Scale=1.1]


% ---------- Packages ----------
\usepackage{geometry}
\usepackage{xcolor}
\usepackage{pagecolor}
\usepackage{tikz}
\usepackage{hyperref}
\usepackage[most]{tcolorbox}
\tcbuselibrary{skins,breakable,listings}
\usepackage{titlesec}
\usepackage{fancyhdr}
\usepackage{listings}
\usepackage[ngerman]{babel}

% ---------- Layout ----------
\geometry{left=26mm,right=26mm,top=28mm,bottom=30mm}
\pagestyle{fancy}
\fancyhf{}
\renewcommand{\headrulewidth}{0pt}
\fancyhead[L]{\textsf{\small PRIN — Q3}}
\fancyhead[R]{\textsf{\small Ferdinand-Braun Schule}}
\fancyfoot[C]{\scriptsize Leistungskurs Praktische Informatik Q3 \; • \; \thepage}



% -------------------- Farben (fein abgestuft, subtiler Fluss) --------------------
\definecolor{PageBG}{RGB}{17,11,31}
\definecolor{TextCream}{RGB}{246,241,234}

\definecolor{AccentBlue}{RGB}{76,104,150}
\definecolor{AccentViolet}{RGB}{105,77,145}

\definecolor{AccentMagenta}{RGB}{140,60,120}

\definecolor{AccentHint}{RGB}{170,40,60}

\definecolor{AccentRed}{RGB}{120,10,25}

\definecolor{BoxBackground}{RGB}{15,9,27}
\definecolor{MarginalGray}{RGB}{150,150,160}

% ---------- Terminal-Farben ----------
\definecolor{TermBG}{RGB}{10,12,14}
\definecolor{TermText}{RGB}{236,240,241}
\definecolor{MatrixGreen}{RGB}{57,255,20}
\definecolor{MatrixGreenDim}{RGB}{35,180,12}
\definecolor{MatrixGreenDeep}{RGB}{18,110,6}
\definecolor{Steel}{RGB}{80,90,100}
\definecolor{AccentWarn}{RGB}{200,40,40}

%\pagecolor{TermBG}
\pagecolor{black!93}
\color{TermText}

% --- NEUE, FESTE FARBEN FÜR JEDE UMGEBUNG ---
\definecolor{aufgabeColor}{RGB}{57,255,20}
\definecolor{loesungColor}{RGB}{230,120,35}
\definecolor{theoremColor}{RGB}{76,104,150}
\definecolor{beispielColor}{RGB}{105,77,145}
\definecolor{infoboxColor}{RGB}{170,40,60}
\definecolor{terminalColor}{RGB}{120,10,25}

% ================= tcolorbox Basis-Stil (Terminal-Look) =================
\tcbset{
	termbase/.style={
		enhanced,
		breakable,
		boxrule=0.9pt,
		colback=TermBG,
		colupper=TermText,
		arc=1mm,
		boxsep=5pt,
		left=14pt,right=14pt,top=12pt,bottom=12pt,
		before skip=8pt, after skip=8pt,
		attach boxed title to top left={yshift=-0.1mm-\tcboxedtitleheight/2, xshift=9mm},
		boxed title style={
			arc=1mm, left=6pt,right=6pt,top=3pt,bottom=3pt, boxrule=0pt
		},
		fonttitle=\sffamily\bfseries\small,
		title after break=\vspace{4pt}
	}
}

% ================= Umgebung-Implementierungen (mit festen Farben) =================
% Wir verwenden eigene LaTeX-Zähler (statt tcolorbox auto counter), damit wir
% vorher \nextboxcolors aufrufen können und die Palette-Makros gesetzt sind.

% ----- Theorem -----
\newcounter{theorem}[section]
\renewcommand{\thetheorem}{\thesection.\arabic{theorem}}
\newenvironment{theorem}[1]{%
	\refstepcounter{theorem}%
	\begin{tcolorbox}[termbase,
		colframe=theoremColor!50!black,
		boxed title style={interior style={left color=theoremColor, right color=theoremColor!70!black}},
		title={Theorem~\thetheorem: #1}]%
	}{\end{tcolorbox}}

% ----- Beispiel -----
\newcounter{beispiel}[section]
\renewcommand{\thebeispiel}{\thesection.\arabic{beispiel}}
\newenvironment{beispiel}[1]{%
	\refstepcounter{beispiel}%
	\begin{tcolorbox}[termbase,
		colframe=beispielColor!50!black,
		boxed title style={interior style={left color=beispielColor, right color=beispielColor!70!black}},
		title={Beispiel~\thebeispiel: #1}]%
	}{\end{tcolorbox}}

% ----- Aufgabe -----
\newcounter{aufgabe}[section]
\renewcommand{\theaufgabe}{\thesection.\arabic{aufgabe}}
\newenvironment{aufgabe}[1]{%
	\refstepcounter{aufgabe}%
	\begin{tcolorbox}[termbase,
		colframe=aufgabeColor!50!black,
		boxed title style={interior style={left color=aufgabeColor, right color=aufgabeColor!70!black}},
		title={Aufgabe~\theaufgabe: #1}]%
	}{\end{tcolorbox}}

% ----- Lösung (verwendet eigenen Zähler, kann optional an Aufgabe gekoppelt werden) -----
\newcounter{loesung}[section]
\renewcommand{\theloesung}{\thesection.\arabic{loesung}}
\newenvironment{loesung}[1]{%
	\refstepcounter{loesung}%
	\begin{tcolorbox}[termbase,
		colframe=loesungColor!50!black,
		boxed title style={interior style={left color=loesungColor, right color=loesungColor!70!black}},
		title={Lösung~\theloesung: #1}]%
	}{\end{tcolorbox}}

% ----- Hinweis / Info -----
\newenvironment{infobox}{%
	\begin{tcolorbox}[termbase,
		colframe=infoboxColor!50!black,
		boxed title style={interior style={left color=infoboxColor, right color=infoboxColor!70!black}},
		title={Hinweis}]%
	}{\end{tcolorbox}}

% ----- Terminal (Listing only) -----
\newenvironment{terminal}[1]{%
	\begin{tcolorbox}[termbase,
		colframe=terminalColor!50!black,
		boxed title style={interior style={left color=terminalColor, right color=terminalColor!70!black}},
		title={Terminal: #1},
		listing only,
		listing options={
			basicstyle=\ttfamily\small, columns=fullflexible,
			showstringspaces=false, tabsize=2, breaklines=true,
			backgroundcolor=\color{TermBG},
			keywordstyle=\color{MatrixGreen},
			commentstyle=\color{Steel},
			stringstyle=\color{MatrixGreenDim},
			frame=none, numbersep=8pt, numbers=none
		}]%
	}{\end{tcolorbox}}

% ================= Sonstiges =================
\lstdefinestyle{darkterm}{
	basicstyle=\ttfamily\small, columns=fullflexible,
	showstringspaces=false, tabsize=2, breaklines=true,
	backgroundcolor=\color{TermBG},
	keywordstyle=\color{MatrixGreen},
	commentstyle=\color{Steel},
	stringstyle=\color{MatrixGreenDim},
	frame=single, rulecolor=\color{MatrixGreenDeep!80!black},
	xleftmargin=0pt, numbersep=8pt, numbers=left,
	numberstyle=\tiny\color{Steel}
}
\lstset{style=darkterm}

\newcommand{\lessondate}[1]{\noindent\hfill\textcolor{Steel}{\textsc{#1}}\\[6pt]}


% ==================== Feines Titelblatt ====================
\newcommand{\MakeArtTitle}[4]{%
	\begin{titlepage}
		\vspace*{18mm}
		\begin{center}
			\vspace{12mm}
			{\huge\color{TextCream} #1 \par}
			\vspace{6mm}
			{\Large\itshape\color{AccentBlue!50} #2 \par}
			\vspace{10mm}
			{\Large\scshape\color{TextCream} #3 \par}
			\vspace{6mm}
			{\small\color{MarginalGray} #4 \par}
			\vfill
			{\small\color{MarginalGray} \today \par}
		\end{center}
	\end{titlepage}
}



% ================= Dokumentbeginn =================
\begin{document}
	
	% Titelblatt
	\MakeArtTitle{Leistungskurs Praktische Q3 Hessen}{Skript}{Shamsher Singh Kalsi}{Berufliches Gymnasium — Ferdinand-Braun Schule \\ Kursleiter: Herr Sebastian Stolz}
	
	\tableofcontents
	\bigskip
	
	\newpage
	
	\section{Einführung}
	\lessondate{05.09.2025}
	
	\begin{aufgabe}{Serielle Kommunikation}
		\begin{enumerate}
			\item Beantworte folgende Fragen schriftlich:
			\begin{itemize}
				\item An welcher Stelle spielt die serielle Kommunikation heutzutage eine Rolle?
				\item Erkläre die Begriffe: Startbit, Datenbit, Stoppbit.
				\item Ein PC sendet den Buchstaben 'A' mit 1 Startbit, 8 Datenbits, 1 Stoppbit. Skizziere das resultierende Bitmuster
				\item Welche Parameter müssen Sender und Empfänger bei RS232 vorab gemeinsam einstellen?
				\item Warum reicht ein einziger Draht für die Übertragung?
			\end{itemize}
			\item Aufgabe 2 – Serielle Kommunikation (RS232) in Java mit Hilfe eines Emulators
			\begin{itemize}
				\item Warum brauchen Sender und Empfänger die gleiche Baudrate?
				\item Was passiert, wenn der Sender schneller schreibt als der Empfänger lesen kann? Welche Lösungen gibt es für dieses Problem?
			\end{itemize}
		\end{enumerate}
	\end{aufgabe}
	
	\newpage
	
	\begin{loesung}{Serielle Kommunikation}
		\begin{enumerate}
			\item Beantworte folgende Fragen schriftlich:
			\begin{itemize}
				\item Serielle Kommunikation wird heute noch in **eingebetteten Systemen**, **Industrieanlagen**, **Messgeräten** und auch beim **Serverzugriff** (Konsolenport) genutzt, da sie einfach, robust und für kurze Distanzen ausreichend ist.
				
				\item \textbf{Startbit:} signalisiert den Beginn eines Zeichens (logisch 0) und dient zur Synchronisation. 
				\textbf{Datenbits:} eigentliche Nutzinformation, meist 8 Bit, LSB zuerst.
				\textbf{Stoppbit:} beendet die Übertragung (logisch 1), die Leitung geht in den Idle-Zustand.
				
				\item Beispiel: Der Buchstabe \texttt{A} hat den ASCII-Wert \texttt{0x41} = \texttt{01000001}. 
				Übertragung (LSB zuerst) mit 1 Startbit und 1 Stopbit: 
				\[
				\underbrace{1}_{\text{Idle}} \;
				\underbrace{0}_{\text{Start}} \;
				\underbrace{1\,0\,0\,0\,0\,0\,1\,0}_{\text{Datenbits (LSB zuerst)}} \;
				\underbrace{1}_{\text{Stop}} \;
				\underbrace{1}_{\text{Idle}}
				\]
				
				\item Sender und Empfänger müssen sich bei RS232 vorab einigen auf: **Baudrate**, **Datenbits**, **Parität**, **Stoppbits**, sowie ggf. Art der **Flusskontrolle**.
				
				\item Ein einzelner Draht genügt pro Richtung, weil Bits nacheinander mit fester Baudrate gesendet werden; Start- und Stoppbits übernehmen die Synchronisation. Für echte Vollduplex-Kommunikation sind jedoch zwei Leitungen (Tx/Rx) plus Masse üblich.
			\end{itemize}
			
			\item Aufgabe 2 – Serielle Kommunikation (RS232) in Java mit Hilfe eines Emulators
			\begin{itemize}
				\item Beide Seiten brauchen dieselbe Baudrate, da es kein separates Taktsignal gibt. Unterschiedliche Baudraten führen zu falscher Bitinterpretation.
				
				\item Wenn der Sender schneller schreibt als der Empfänger liest, läuft dessen Puffer über und Daten gehen verloren. 
				Lösungen: **Hardware-Flowcontrol** (RTS/CTS), **Software-Flowcontrol** (XON/XOFF), größere **FIFO-Puffer** oder Protokolle mit Bestätigung (ACK/NACK).
			\end{itemize}
		\end{enumerate}
	\end{loesung}
	
	\begin{center}
		\begin{tikzpicture}[yscale=0.8, xscale=0.8]
			% Bitfolge: Idle(1), Start(0), Data: 1 0 0 0 0 0 1 0, Stop(1), Idle(1)
			\foreach [count=\n] \b in {1,0,1,0,0,0,0,0,1,0,1,1} {
				\draw (\n-1,0) rectangle ++(1,1) node[midway]{\b};
			}
			% Beschriftung
			\node[below] at (0.5,-0.2) {Idle};
			\node[below] at (1.5,-0.2) {Start};
			\node[below] at (5.5,-0.2) {Datenbits};
			\node[below] at (10.5,-0.2) {Stop};
		\end{tikzpicture}
	\end{center}
	
	\vspace{1cm} % Added vertical space for better readability
	
	\lstset{language=Java, basicstyle=\ttfamily\small, numbers=left, frame=single}
	\begin{lstlisting}
		import com.fazecast.jSerialComm.SerialPort;
		public class Sender {
			public static void main(String[] args) throws Exception {
				SerialPort sp = SerialPort.getCommPort("COM5");
				sp.setBaudRate(9600);
				sp.openPort();
				sp.getOutputStream().write("Hallo, COM6\n".getBytes());
				sp.closePort();
			}
		}
	\end{lstlisting}
	
	\newpage
	
	\lessondate{09.09.2025}
	\begin{aufgabe}{Steuerung eines Mikroprozessors mit der seriellen Schnittstelle}
		\begin{enumerate}
			\item Schreibe ein (Python-)Programm, welches einen selbst gewählten Sensor auf einem Raspberry-PI oder einem Arduino steuert.
			\item Erweitere das Programm so, dass es über eine serielle Schnittstelle angesprochen werden kann. Emuliere die serielle Schnittstelle mit Hilfe von Software oder nutze einen RS232/TTL Wandler mit MAX3232
			\item Nutze die Klasse "Serial" aus dem Moodle-Kurs, um von einem Laptop oder PC mit einem Java-Programm über die serielle Schnittstelle den Sensor zu steuern.
		\end{enumerate}
	\end{aufgabe}
	
	
	\begin{loesung}{Steuerung eines Mikroprozessors mit der seriellen Schnittstelle}
		\begin{enumerate}
			\item Schreibe ein Python-Programm, das einen Sensor bzw. ein Aktor-Device auf einem Raspberry Pi steuert.
			
			\begin{lstlisting}[language=Python, caption={Raspberry Pi: LED-Steuerung + serielle Steuerung (pySerial + gpiozero)}]
				# requirements: gpiozero, pyserial
				# Ein Python-Skript, das eine LED steuert und auf Befehle über die serielle Schnittstelle hört.
				from gpiozero import LED
				import serial
				import time
				
				# Pin-Nummer der LED und Details zur seriellen Schnittstelle.
				LED_PIN = 17
				SERIAL_PORT = '/dev/ttyUSB0'
				BAUD = 9600
				
				led = LED(LED_PIN)
				ser = serial.Serial(SERIAL_PORT, BAUD, timeout=1)
				
				# Verarbeitet die empfangenen Befehle.
				def handle_line(line):
				line = line.strip().upper()
				if line == 'LED ON':
				led.on()
				ser.write(b'OK\n')
				elif line == 'LED OFF':
				led.off()
				ser.write(b'OK\n')
				elif line == 'STATUS':
				ser.write(b'ON\n' if led.is_lit else b'OFF\n')
				else:
				ser.write(b'ERR Unknown command\n')
				
				# Hauptschleife zum Empfangen von Daten.
				try:
				while True:
				raw = ser.readline()
				if raw:
				handle_line(raw.decode('utf-8'))
				time.sleep(0.01)
				finally:
				ser.close()
				led.off()
			\end{lstlisting}
			
			
			\item Ein Arduino-Beispiel, das serielle Kommandos entgegennimmt und einen digitalen Pin steuert:
			\begin{lstlisting}[language=C, caption={Arduino: Serial command handler}]
				// Ein Arduino-Sketch, der auf serielle Kommandos reagiert.
				const int LED_PIN = 13;
				
				void setup() {
					pinMode(LED_PIN, OUTPUT);
					Serial.begin(9600); // Startet die serielle Kommunikation.
				}
				
				void loop() {
					if (Serial.available() > 0) { // Prüft, ob Daten verfügbar sind.
						String cmd = Serial.readStringUntil('\n'); // Liest eine Zeile.
						cmd.trim();
						if (cmd == "LED ON") {
							digitalWrite(LED_PIN, HIGH);
							Serial.println("OK");
						} else if (cmd == "LED OFF") {
							digitalWrite(LED_PIN, LOW);
							Serial.println("OK");
						} else if (cmd == "STATUS") {
							Serial.println(digitalRead(LED_PIN) ? "ON" : "OFF");
						} else {
							Serial.println("ERR");
						}
					}
				}
			\end{lstlisting}
			
			Die Arduino-API stellt `Serial.available()` und `Serial.read()` / `readStringUntil()` bereit; `available()` gibt die Anzahl bereits empfangener Bytes an, `read()` liefert das nächste Byte oder -1, wenn nichts da ist — das ist der übliche Pattern für nicht-blockierende Abfragen auf Arduino.
			
			\item Java-Client mit der Klasse \texttt{Serial} (wie in deinem Moodle-Skript beschrieben). Das Beispiel öffnet den Port, schickt einen String, liest eine Antwortzeile und parst eine Zahl mit \texttt{Double.parseDouble(...)}:
			\begin{lstlisting}[language=Java, caption={Java: Steuerprogramm (Nutzungsbeispiel der in der Aufgabenstellung beschriebenen Serial-Klasse)}]
				// Ein Java-Client zum Steuern des Mikroprozessors.
				public class SerialController {
					
					public static void main(String[] args) {
						Serial s = new Serial("COM5", 9600, 8, 1, 0); 
						
						if (!s.open()) {
							System.err.println("Port konnte nicht geöffnet werden");
							return;
						}
						
						// Sendet einen Befehl und liest die Antwort.
						s.write("STATUS\n");
						String reply = s.readLine();
						System.out.println("Reply: " + reply);
						
						// Versucht, die Antwort in eine Zahl umzuwandeln.
						try {
							double val = Double.parseDouble(reply.trim());
							System.out.println("Parsed value: " + val);
						} catch (NumberFormatException e) {
							System.out.println("Keine Zahl: " + reply);
						}
						
						s.close();
					}
				}
			\end{lstlisting}
		\end{enumerate}
	\end{loesung}
	
	\newpage

	\lessondate{17.09.2025}\\

	\begin{aufgabe}{Prüfverfahren}
		\begin{enumerate}
			\item Berechne die Ergebnisse der Prüfverfahren Paritätsbit, Prüfsumme (mod4, mod8) und XOR-Prüfsumme (mit zwei Bitfolgen):
			\begin{itemize}
				\item 1010101111010010
				\item 1111111111111111
				\item 0010110001010000
			\end{itemize}
			\item Bei der Übertragung von Bitfolgen können folgende Fehler auftreten:
			\begin{itemize}
				\item ein Bit wird negiert
				\item zwei benachbarte Bits werden ausgetauscht,
				\item zwei Bits werden negiert
			\end{itemize}
			Bewerte die Möglichkeiten der Fehlererkennung durch die einzelnen
			Prüfverfahren
			\item Bücher sind eindeutig durch eine ISBN identifiziert. Die Prüfnummer ist ein Zeichen zwischen 0 und X, für 10, das am Ende angehängt wird. Die Prüfnummer wird als gewichtete Prüfsumme berechnet: Die Tribute von Panem 1 hat die ISBN-Nummer 384150134. Die einzelnen Stellen werden von links nach rechts addiert, mit dem Gewicht der eigenen
			Stelle in der ISBN-Nummer. Die Prüfnummer muss so gewählt werden, dass die gewichtete Prüfsumme mod11 gerechnet 0 ergibt. \(()3 \cdot 10 + 8 \cdot 9 + 4 \cdot 8 + 1 \cdot 7 + 5 \cdot 6 + 0 \cdot 5 + 1 \cdot 4 + 3 \cdot 3 + 4 \cdot 2 + ?) \mod{11} = 192\mod{11} = 5\) Rechnet man mit ? = 6 ergibt sich \(198 \mod{11} = 0\)
			\begin{itemize}
				\item Berechne die Prüfnummer für Die Tribute von Panem 2 mit der ISBN 378913219? 
			\end{itemize}
		\end{enumerate}
		
	\end{aufgabe}
	
	

	\newpage
	
	
	\begin{theorem}{Fundamentaler Satz}
		Inhalt des Theorems ...
	\end{theorem}
	
	\begin{beispiel}{Erstes Beispiel}
		Dieses Beispiel illustriert den Satz.
	\end{beispiel}
	
	\begin{aufgabe}{Rechenaufgabe}
		Bearbeite folgende Aufgabe ...
	\end{aufgabe}
	
	\begin{loesung}{zur Aufgabe}
		Hier die Lösungsschritte ...
	\end{loesung}
	
	\begin{infobox}
		Ein kurzer Hinweis.
	\end{infobox}
	
	\begin{terminal}{Beispielcode}
		echo "Hallo Welt"
		ls -la
	\end{terminal}
	
\end{document}