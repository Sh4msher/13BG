% ============== Dark-Mode Skript mit bereinigten Boxen (XeLaTeX / LuaLaTeX) ==============
\documentclass[11pt,a4paper,oneside]{article}

% -------------------- Engine & Fonts --------------------
\usepackage{fontspec}
\setmainfont{Libertinus Serif}
\setsansfont{Libertinus Sans}
\usepackage{unicode-math}
\setmathfont{Libertinus Math}

% -------------------- Pakete --------------------
\usepackage[utf8]{inputenc}
%\usepackage{amsmath} % Mathematik-Pakete
%\usepackage{amsfonts}
%\usepackage{mathptmx} %times new roman
%\usepackage[T1]{fontenc} %vollen Zeichensatz ]
%\usepackage{amssymb}
%\usepackage{tcolorbox}
\usepackage[ngerman]{babel} 
\usepackage{graphicx}
\usepackage{microtype} %besserer Randausgleich
\usepackage{footnote}
\usepackage{blindtext}
\usepackage{etoolbox}
%\usepackage{makeidx}
%\usepackage{dsfont}
\usepackage{lettrine}%\usepackage{geometry}
%\usepackage{xcolor} % Für verschiedene Farben
\newcommand{\ricardo}[1]{\colorbox{ForestGreen}{\color{white}   \textsf{\textbf{Ricardo}}} \textcolor{ForestGreen}{#1}}
\usepackage[pdfborderstyle={/S/U/W 1}]{hyperref} % Für interaktive Refernzierung im PDF
\usepackage{csquotes}
\usepackage{acro}
\usepackage{hyperref} % Für interaktive Refernzierung im PDF
\usepackage[onehalfspacing]{setspace}%Zeilenabstand 1.5
% \usepackage{picins} % Das Umfließen einer Grafik im Text kann mit dem Paket PicIns erreicht werden.
%\usepackage{fontspec} 
\usepackage{mathtools}
%\usepackage[utf8]{inputenc}
\usepackage[ngerman]{babel}
\usepackage[top=3cm, bottom=3cm, left=2.5cm, right=2.5cm]{geometry}
\usepackage{bibgerm}
\usepackage{tabularx}
\usepackage{adjustbox}
\usepackage{cite}
\usepackage{blindtext}
\usepackage{epsfig}
\usepackage{longtable}
%\usepackage{showframe}
\usepackage{dcolumn}%benötigt für stargaze
\usepackage{here}%lädt das Paket zum Erzwinge n der Grafikposition
\usepackage{floatflt}%Bilder im Fließtext
%\usepackage{fontspec}
%\usepackage{fontenc}
\usepackage{dsfont}
%\setsansfont[Ligatures=TeX]{Arial}
%\renewcommand{\familydefault}{\sfdefault}
%\usepackage{times}
\usepackage{graphicx}
\usepackage{epstopdf}
\usepackage{physics}
\usepackage[siunitx]{circuitikz} %[symbols]
\usepackage{xcolor}
\usepackage{listings}

\usepackage{lipsum}

\usepackage[utf8]{inputenc}
\usepackage[T1]{fontenc}
\usepackage[ngerman]{babel}
\usepackage{amsmath}
\usepackage{tocloft}
\renewcommand{\cftsecleader}{\cftdotfill{\cftdotsep}} % Punkte zwischen Abschnittsnummern und -titeln



\usepackage{pgfplots}
\pgfplotsset{compat=1.18} % aktuelle Version, sonst 1.17 oder 1.16
\usepackage{paralist}
\usepackage{tasks}
\usepackage{acro}
\usepackage{microtype}
\usepackage{geometry}
\usepackage{titlesec}
\usepackage{fancyhdr}
\usepackage{xcolor}
\usepackage{pagecolor}
\usepackage{tikz}
\usetikzlibrary{shadows,calc}
\usepackage[most]{tcolorbox}
\tcbuselibrary{skins,breakable,theorems}
\usepackage{enumitem}
\usepackage{caption}
\usepackage{everypage}

% -------------------- Layout --------------------
\geometry{
	left=28mm, right=28mm, top=28mm, bottom=28mm,
	marginparwidth=36mm, marginparsep=6mm
}

% -------------------- Farben (fein abgestuft, subtiler Fluss) --------------------
\definecolor{PageBG}{RGB}{17,11,31}             % sehr dunkler Hintergrund
\definecolor{TextCream}{RGB}{246,241,234}       % creme / sehr hell

\definecolor{AccentBlue}{RGB}{76,104,150}       % akademisches Blau (unverändert)
\definecolor{AccentViolet}{RGB}{105,77,145}     % seriöses Violett (unverändert)

% Hint: mehr ins violett/pinke ziehen (Hauch von Fuchsia)
\definecolor{AccentMagenta}{RGB}{140,60,120}       % violett-pink (sanfter Übergang)

% Magenta: rötlicher, dichter, aber nicht grell
\definecolor{AccentHint}{RGB}{170,40,60}     % rötliches Magenta (mehr Rotanteil)

% Elegantes Rot: orientiert sich am bisherigen "AccentHint" (tiefes Weinrot)
\definecolor{AccentRed}{RGB}{120,10,25}         % elegantes Weinrot / tiefes Rot

\definecolor{BoxBackground}{RGB}{15,9,27}       % Box-Hintergrund
\definecolor{MarginalGray}{RGB}{150,150,160}    % Rand-Datum

\pagecolor{PageBG}
\color{TextCream}


% -------------------- Kopf / Fuß --------------------
\pagestyle{fancy}
\fancyhf{}
\renewcommand{\headrulewidth}{0pt}
\setlength{\headheight}{14pt}
\fancyfoot[C]{%
	\vspace{6pt}%
	{\scriptsize\scshape Ferdinand-Braun Schule \quad • \quad Leistungskurs Mathematik Q3 \quad • \quad \thepage}%
	\begin{tikzpicture}[remember picture,overlay]
		\draw[line width=0.9pt,color=AccentViolet!80!black] ($(current page.south west)+(28mm,22mm)$) -- ($(current page.south east)+(-28mm,22mm)$);
	\end{tikzpicture}%
}

% -------------------- Titel-Styles --------------------
\titleformat{\section}
{\normalfont\large\bfseries\color{AccentBlue!33}}
{\thesection}
{1em}
{}
\titleformat{\subsection}{\normalfont\normalsize\bfseries\color{AccentViolet!33}}{\thesubsection}{0.8em}{}
\titleformat{\subsubsection}{\normalfont\normalsize\bfseries\color{AccentMagenta!33}}{\thesubsubsection}{0.6em}{}

% ========================= TCBOX BASIS-STIL (bereinigt) =========================
\tcbset{
	mybase/.style={
		enhanced,
		breakable,
		%drop shadow, 
		%shadow={1mm}{-1mm}{0mm}{black!50!white},
		boxrule=0.9pt,
		colframe=black!20,
		colback=BoxBackground,
		colupper=TextCream,          % Textfarbe innerhalb der Box
		arc=2mm,
		boxsep=5pt,
		left=15pt,right=15pt,top=15pt,bottom=15pt,
		before skip=8pt, after skip=8pt,
		attach boxed title to top left={yshift=-0.25mm-\tcboxedtitleheight/2, xshift=10mm},
		boxed title style={
			arc=4mm,
			left=6pt,right=6pt,top=3pt,bottom=3pt,
			boxrule=0pt           % kein zusätzlicher Rahmen für den Titelsticker
			% hier KEIN colback setzen — colbacktitle wird pro Box gesetzt
		},
		fonttitle=\sffamily\bfseries\small,
		title after break=\vspace{4pt}
	}
}

% -------------------- THEOREM (theo) --------------------
\newtcolorbox[auto counter,number within=section]{theo}[2][]{%
	mybase,
	colframe = AccentBlue!75!black,  %AccentViolet!75!black,
	colbacktitle = AccentBlue!85!black, %AccentViolet!85!black,
	coltitle = TextCream,
	title = {Theorem~\thetcbcounter: #2},
	#1
}

% -------------------- BEISPIEL (exem) --------------------
\newtcolorbox[auto counter,number within=section]{exem}[2][]{%
	mybase,
	colframe = AccentViolet!75!black, %AccentBlue!75!black,
	colbacktitle = AccentViolet!85!black,%AccentBlue!85!black,
	coltitle = TextCream,
	title = {Beispiel~\thetcbcounter: #2},
	#1
}

% -------------------- AUFGABE (aufgabe) --------------------
\newtcolorbox[auto counter,number within=section]{aufgabe}[2][]{%
	mybase,
	colframe = AccentMagenta!75!black, %AccentMagenta!75!black,
	colbacktitle = AccentMagenta!85!black, %AccentMagenta!85!black,
	coltitle = TextCream,
	title = {Aufgabe~\thetcbcounter: #2},
	#1
}

% -------------------- LÖSUNG (loesung) - zu Aufgabe passend --------------------
\newtcolorbox[use counter from=aufgabe]{loesung}[2][]{%
	mybase,
	colframe = AccentHint!75!black, %AccentMagenta!60!black,
	colbacktitle = AccentHint!85!black,%AccentMagenta!72!black,
	coltitle = TextCream,
	title = {Lösung~\thetcbcounter: #2},
	#1
}

% -------------------- Hinweis-Box --------------------
\newtcolorbox{infobox}[1][]{%
	mybase,
	colframe = AccentRed!75!black,
	colbacktitle = AccentRed!85!black,
	coltitle = TextCream,
	title = {Hinweis},
	#1
}

% ==================== Datum in rechter Margin (jede Seite) ====================
%\newcommand{\SetSideDateText}[1]{%
%	\gdef\@sidedatetext{#1}%
%	\AddEverypageHook{%
%		\begin{tikzpicture}[remember picture,overlay]
%			\node[anchor=north east,inner sep=0pt] at ($(current page.north east)+(-18mm,-14mm)$) {%
%				\parbox{32mm}{\raggedleft\small\sffamily\color{MarginalGray}\@sidedatetext}%
%			};
%		\end{tikzpicture}%
%	}%
%}
%\makeatletter \def\@sidedatetext{} \makeatother

% Makro für das Datum am rechten Rand
\newcommand{\lessondate}[1]{
	\noindent\hfill\textcolor{gray}{\textsc{#1}} \\
	\vspace{0.5cm}
}



% ==================== Feines Titelblatt ====================
\newcommand{\MakeArtTitle}[4]{%
	\begin{titlepage}
		\vspace*{18mm}
		\begin{center}
			%\begin{tikzpicture}
			%	\fill[AccentViolet!85!black] (0,0) circle (1.2cm);
			%	\fill[PageBG] (0.15,0.15) circle (0.95cm);
			%	\draw[line width=1pt,color=AccentBlue] (0,0) circle (1.35cm);
			%	\node[white] at (0,0) {\sffamily\bfseries\Large M};
			%\end{tikzpicture}
			\vspace{12mm}
			%{\Huge\bfseries\sffamily\color{TextCream} #1 \par}
			{\huge\color{TextCream} #1 \par}
			\vspace{6mm}
			{\Large\itshape\color{AccentBlue!50} #2 \par}
			\vspace{10mm}
			{\Large\scshape\color{TextCream} #3 \par}
			\vspace{6mm}
			{\small\color{MarginalGray} #4 \par}
			%\vfill
			\vspace{5mm}
			{\small\color{MarginalGray} \today \par}
			%\vspace{12mm}
			%\begin{tikzpicture}
			%	\draw[line width=1.2pt,color=AccentViolet!80!black] (-6,0) -- (6,0);
			%	\draw[line width=0.45pt,color=AccentBlue!60!black] (-6,-0.3) -- (6,-0.3);
			%\end{tikzpicture}
		\end{center}
		\vspace{7.5cm}
		\centering
		\includegraphics[width=0.75\textwidth]{2.png} % Logo einfügen (Pfad anpassen)
	\end{titlepage}
}

% ==================== Dokumentbeginn ====================
\begin{document}
	
	% Titelblatt
	\MakeArtTitle{
		Leistungskurs Mathematik Q3 Hessen}
		{Stochastik Skript}
		{Shamsher Singh Kalsi}
		{Berufliches Gymnasium — Ferdinand-Braun Schule \\ Kursleiter: Herr Thorsten Farnungen}
	
	\tableofcontents
	\bigskip
	\clearpage
	
	
	\section{Einleitung}
	\lessondate{17.08.2025}\\
	Dieses Skript dient als Fortsetzung von der Q2. Hierbei werden nur thematisch theoretische Unterrichtsinhalte notiert, wobei die Aufgaben und Übungen hauptsächlich in Obsidian bearbeitet werden, um den wahnsinnigen Dokumentationsaufwand zu reduzieren. 
	
	\subsection{Leistungsbewertung und Klausurplanung in der Q3}
	\lessondate{20.08.2025}\\
	Im Verlauf des Schuljahres werden in diesem Kurs drei Klausuren geschrieben.
	Die zweite Klausur wird als Abiturklausur unter authentischen Bedingungen angesetzt, was eine Bearbeitungszeit von fünf Zeitstunden umfasst. Da der bis zu diesem Zeitpunkt behandelte abiturrelevante Stoff der Q3 ausschließlich die Stochastik abdeckt, wäre eine fünfstündige Prüfung allein zu diesem Thema für die Schülerinnen und Schüler eine unzumutbare Belastung. 
	Aus diesem Grund wird der hilfsmittelfreie Teil dieser Klausur auch Aufgaben aus den Qualifikationsphasen Q1 (Analysis) und Q2 (Analytische Geometrie/Lineare Algebra) beinhalten, um eine angemessene Themenbreite zu gewährleisten.
	Die erste Klausur ist für den Zeitraum vor den Herbstferien vorgesehen.
	
	\subsubsection{Randbemerkungen}

	%Zufallsexperimente, bei denen alle möglichen Ergebnisse \emph{gleich wahrscheinlich} angenommen werden. Dies ist kein naturwissenschaftliches Faktum, sondern ein mathematisches Modell unter Symmetrieannahmen.
	\begin{theo}{Laplace Experiment}
		Ein Zufallsexperiment mit endlicher Ergebnismenge $\Omega$ heißt
		\emph{Laplace-Experiment}, wenn für jedes Elementarereignis $\omega \in \Omega$
		gilt:
		\[
		P(\{\omega\}) = \frac{1}{|\Omega|}.
		\]
		Die Wahrscheinlichkeit eines Ereignisses $E \subseteq \Omega$ ist dann
		\[
		P(E) = \frac{|E|}{|\Omega|}.
		\]
	\end{theo}
	
	\begin{exem}
		Das Werfen eines idealen Würfels: $\Omega = \{1,2,3,4,5,6\}$,
		jedes Ergebnis hat Wahrscheinlichkeit $\tfrac{1}{6}$.
		Für das Ereignis $E=\{\text{gerade Zahl}\}$ gilt
		$P(E)=\tfrac{3}{6}=\tfrac{1}{2}$.
	\end{exem}
	
	\newpage
	
	\section{Grundlegende Begriffe der Stochastik}
	
	\begin{aufgabe}{Check in Kapitel 1}
		\begin{enumerate}
			\item Jan hat 20-mal in eine Lostrommel hineingegriffen und dabei 18 Nieten gezogen. 
			\begin{itemize}
				\item Berechnen Sie die relative Häufigkeit für den Gewinn als Bruch und in Prozent
				\item Jana erreichte bei 12 Ziehungen die Gewinnquote 25\%. Berechnen Sie die absolute und die relative Häufigkeit. 
			\end{itemize}
		\item Bei der Bundestagswahl 2013 haben sich $71.5\%$ der 62 Mio. Wahlberechtigten an der Wahl beteiligt. Die Stimmenverteilung für die einzelnen Parteien ist in Fig. 1 Dargstellt. 
		\begin{itemize}
			\item Geben Sie die Anteile der Stimmenverteilung als Bruch und Dezimalzahl an
			\item Berechnen Sie, wie groß der Stimmenanteil der einzelnen Parteien bezogen auf alle 62 Mio. Wahlberechtigten ist.
		\end{itemize}
		\item Begründen Sie welche Situation ein Laplace-Experiment darstellt
		\begin{itemize}
			\item Sie fragen einen Lehrer, an welchem Wochentag er sein Auto gewaschen hat 
			\item Sie ziehen ein Los aus einem Loseimer mit 120 Losen 
			\item Sie beobachten, ob der nächste Plattfuß an iuhrem Fahrrad vorne oder hinten auftritt
		\end{itemize}
		\item Berechnen Sie den Mittelwert der folgenden Zahlen: \\
		$2,5 \qquad 6,3 \qquad 1,9 \qquad 10, 0 \qquad 2,8 \qquad 5,6 \qquad 5,1 \qquad 7,8$
		\end{enumerate}
	\end{aufgabe}
	
	\newpage
	
	\begin{loesung}{}
		\subsection*{Aufgabe 1}
		\begin{align*}
			h = \frac{H}{n}, &\qquad n = 20, \qquad H = 20 - 18 = 2 \\
			h  &= \frac{2}{20} = \boxed{ \frac{1}{10} = 0.1 = 10\%}
		\end{align*}
		Jana's relative häufige Gewinnquote Beträgt $25\%$, sodass $75\% \lor \frac{3}{4}$ Nieten sein müssen. Es gilt; 
		\[
		h_n(A) = \frac{H_n(A)}{n},
		\]
		wobei $h_n(A)$ die relative und $H_n(A)$ die absolute Häufigkeit sind. 
		\begin{align*}
			H &= h \cdot n\\
			H &= \boxed{0.25 \cdot 12 = 3}  
		\end{align*}
		\subsection*{Aufgabe 2}		
		\subsubsection*{Stimmenverteilung}
		\begin{inparaitem}
			\item [\textbf{CDU}] : 34.1 \% $= \frac{34.1}{100} = 0.341$ |
			\item [\textbf{CSU}] : 7.4\% $= \frac{7.4}{100} = 0.074$ |
			\item [\textbf{SPD}] : 25.7 \% $= \frac{25.7}{100} = 0.257$ |
			\item [\textbf{FDP}] : 4.8\% $= \frac{4,8}{100} = 0.048$ |
			\item [\textbf{Die Linke}] : 8.6\%  $= \frac{8.6}{100} = 0.086$ |
			\item [\textbf{Die Grünen}] : 8.4\% $= \frac{8.4}{100} = 0.084$ |
			\item [\textbf{sonstige}] : 10.9 \%$= \frac{10.9}{100} = 0.109$ |
		\end{inparaitem}\\ 
		\subsubsection*{Stimmenanteil}
		\[
		\text{Wähler} = 62.000.000 \cdot 0.715 = 44.330.000
		\]
		\begin{itemize}[leftmargin= 4cm, font=\bfseries]
			\item [\textbf{CDU}] : 34.1 \% $= 0.341 \cdot 44.330.000 \approx \boxed{15.112.000}$ 
			\item [\textbf{CSU}] : 7.4\% $= 0.074 \cdot 44.330.000 \approx \boxed{3.280.420} $ 
			\item [\textbf{SPD}] : 25.7 \% $= 0.257 \cdot 44.330.000 \approx \boxed{11.392.810}$ 
			\item [\textbf{FDP}] : 4.8\% $= 0.048 \cdot 44.330.000 \approx \boxed{2.127.840}$ 
			\item [\textbf{Linke}] : 8.6\%  $=0.086 \cdot 44.330.000 \approx \boxed{3.812.380}$ 
			\item [\textbf{Grünen}] : 8.4\% $= 0.084 \cdot 44.330.000 \approx \boxed{3.723.720}$ 
			\item [\textbf{sonstige}] : 10.9 \%$= 0.109 \cdot 44.330.000 \approx \boxed{4.831.970}$ 
		\end{itemize}		
	\end{loesung}
	
	\newpage
	
	\begin{loesung}{}
		\subsection*{Aufgabe 3}
		\begin{enumerate}
			\item Kein Laplace-Experiment, da der Lehrer nicht mit gleicher Wahrscheinlichkeit an jedem Wochentag sein Auto wäscht; Alltagsroutinen und äußere Zwänge machen die Wahrscheinlichkeiten ungleich.
			\item Laplace-Experiment: Jedes der 120 Lose ist gleich wahrscheinlich gezogen zu werden, sofern alle Lose gleich beschaffen und gut gemischt sind. Dass die Gewinnchancen inhaltlich ungleich verteilt sind (1 Gewinnlos, 119 Nieten), widerspricht dem Laplace-Modell nicht, da sich dieses nur auf die Elementarereignisse (jedes einzelne Los) bezieht.
			\item Kein sauberes Laplace-Experiment, da das Vorderrad physikalisch häufiger betroffen ist (führt, trifft zuerst Hindernisse, andere Belastung). Nur unter starker Modellannahme „beide Räder gleich gefährdet“ könnte man es als Laplace-Experiment ansehen.
		\end{enumerate}
		\subsection*{Aufgabe 4}
		\begin{itemize}
			\item $2 + 5 = 7, \frac{7}{2} = 3,5$
			\item $6 + 3 = 9, \frac{9}{2} = 4.5$
			\item $1 + 9 = 10, \frac{10}{2} = 5$
			\item $10 + 0 = 10, \frac{10}{2} = 5$
			\item $2 + 8 = 10, \frac{10}{2} = 5$
			\item $5 + 6 = 11, \frac{11}{2} = 5.5$
			\item $5 + 1 = 6, \frac{6}{2} = 3$
			\item $7 + 8 = 15, \frac{15}{2}$
		\end{itemize}
		
		\textbf{Oder:} Um den Mittelwert (das arithmetische Mittel) $\bar{x}$ zu berechnen, werden alle Zahlen summiert und die Summe wird durch die Anzahl der Zahlen geteilt.
		\begin{align*}
			\bar{x} &= \frac{2,5 + 6,3 + 1,9 + 10,0 + 2,8 + 5,6 + 5,1 + 7,8}{8} \\
			\bar{x} &= \frac{42,0}{8} \\
			\bar{x} &= \boxed{5,25}
		\end{align*}
	\end{loesung}
	
	\newpage 
	\lessondate{25.08.2025}\\
	
	\begin{aufgabe}{Bearbeiten Sie die Aufgaben 1 bis 4 von Wdh. Statistik}
		\begin{enumerate}
			\item Berechnen Sie den Mittelwert, die Varianz und die Standardabweichung der Liste ${2; 0; 5; 6; 3; 8}$
			\item Von einer Lieferung Fahrradspeichen wurde bei einer Stichprobe die Länge der Speichen (in mm) gemessen: ${269; 274; 269; 268; 272; 270; 269; 270; 268; 271}$. 
			\begin{itemize}
				\item Nenne Sie die bei dieser Erhebung die Grundgesamtheit, den Mermalsträger, das untersuchte Merkmal, den Stichprobenumfang und die Merkmalsausprägungen. 
				\item Berechnen Sie den Mittelwert, die Varianz und die Standardabweichung 
			\end{itemize}
			\item Die Anzahl der Regentage beträgt im langjährigen Mittel für Amsterdam bzw. Rangun: 
			\begin{itemize}
				\item Stellen Sie die Verteilung der Anzahl der Regentage grafisch dar. 
				\item Berechnen Sie für beide Messreihen den Mittelwert und die Standartabweichung
			\end{itemize}
			\item Gegeben ist die nebenstehende relative Häufgikeitsveteilung.
			\begin{itemize}
				\item Beschriften Sie die Achsen passend 
				\item Bestimmen Sie den Mittelwert und die Standardabwichung 
				\item Untersuchen Sie, welche Werte aus b) sich ändern, wenn alle Säulen gleich hoch sind
			\end{itemize}
		\end{enumerate}
	\end{aufgabe}
	
	\newpage
	
	\begin{theo}{Die Bedeutung von Mittelwert, Varianz und Standardabweichung}
		\subsection*{Mittelwert}
		Der Mittelwert gibt sozusagen den Durchschnitt gegebener Daten an. Diesen Berechnet man durch das Addieren aller Elemente und dem Teilen von der Anzahl der Elemente. 
		\[
		\overline{x} = \frac{1}{n} \sum^{n}_{i = 1} x_i 
		\]
		\subsection*{Varianz}
		Die Varianz misst, wie stark die Werte um den Mittelwert streuen. Dazu berechnet man die Abweichungen jedes Wertes vom Mittelwert, quadriert diese (damit Abweichungen nach oben und unten nicht wegfallen) und mittelt sie wieder:
		\[
		s^2 = \frac{1}{n} \sum^{n}_{i = 1} (x_i - \overline{x})^2 
		\]
		\subsection*{Standardabweichung}
		Die Standardabweichung ist die Wurzel der Varianz. Sie gibt die Streuung in derselben Einheit wie die Daten an (praktischer als die quadrierten Werte der Varianz):
		\[
		s = \sqrt{s^2}
		\]
	\end{theo}
	
	\begin{loesung}{Aufgabe 1}
	\begin{enumerate}
		\item \subsection*{Mittelwert}
		\vspace{-4mm}
		\[
			\overline{x} = \frac{2 + 0 + 5 + 6 + 3 + 8}{6} = \boxed{\frac{24}{6} = 4}
		\]
		\item \subsection*{Varianz}
		\vspace{-6mm}
		\begin{align*}
			s^2 &= \frac{1}{n} \sum^{n}_{i = 1} (x_i - \overline{x})^2 = \frac{\sum^{n}_{i = 1} (x_i - \overline{x})^2}{n}\\
			&= \frac{(2 - 4)^2 + (0 - 4)^2 + (5 - 4)^2 + (6 - 4)^2 + (3 - 4)^2 + (8 - 4)^2}{6}\\
			&= \frac{4 + 16 + 1 + 4 + 1 + 16}{6} = \boxed{\frac{42}{6} = 7}
		\end{align*}
	\item \subsection*{Standardabweichung} 
	\[
	s = \sqrt{s^2} \rightarrow \boxed{\sqrt{7} \approx 2.65}
	\]
	\end{enumerate}
	\end{loesung}
	
	\newpage
	
	\begin{loesung}{Aufgabe 2}
		\begin{enumerate}
			\item \subsection*{Aufgabe A}
			\begin{itemize}[left=4cm]
				\item[\textbf{Grundgesamtheit:}] Alle Fahrradspeichen in der gesamten Lieferung 
				\item[\textbf{Merkmalsträger:}] Eine einzelne Fahrradspeiche 
				\item[\textbf{Merkmal:}] Die Länge der Speiche in mm 
				\item[\textbf{Stichprobenumfang:}] Es wurden 10 Speichen gemessen $\rightarrow n = 10$ 
				\item[\textbf{Merkmalausprägungen:}] ${269; 274; 269; 268; 272; 270; 269; 270; 268; 271}$
			\end{itemize}
			
			\item \subsection*{Aufgabe B}
			
			\subsubsection*{Mittelwert}
			\vspace{-4mm}
			\[
			\overline{x} = \frac{269 + 274 + 269 + 268 + 272 + 270 + 269 + 270 + 268 + 271}{10} 
			= \boxed{\frac{2700}{10} = 270}
			\]
			
			\subsubsection*{Varianz}
			\vspace{-6mm}
			\begin{align*}
				s^2 &= \frac{1}{n} \sum^{n}_{i = 1} (x_i - \overline{x})^2 \\[2mm]
				&= \frac{(269-270)^2 + (274-270)^2 + (269-270)^2 + (268-270)^2 + (272-270)^2}{10} \\
				&\quad + \frac{(270-270)^2 + (269-270)^2 + (270-270)^2 + (268-270)^2 + (271-270)^2}{10} \\[2mm]
				&= \frac{1 + 16 + 1 + 4 + 4 + 0 + 1 + 0 + 4 + 1}{10} \\[2mm]
				&= \boxed{\frac{32}{10} = 3.2}
			\end{align*}
			
			\subsubsection*{Standardabweichung}
			\[
			s = \sqrt{s^2} = \sqrt{3.2} \approx \boxed{1.79}
			\]
			
		\end{enumerate}
	\end{loesung}
	
	\newpage
	
	\begin{loesung}{Aufgabe 3}
    \subsection{Aufgabe A}
		\resizebox{\textwidth}{!}{
			\begin{tabular}{|c|c|c|c|c|c|c|c|c|c|c|c|c|}
				\hline
				Monat & Jan. & Feb. & März & April & Mai & Juni & Juli & Aug. & Sept. & Okt. & Nov. & Dez. \\
				\hline
				Amsterdam & 10 & 8 & 11 & 8 & 9 & 9 & 11 & 11 & 10 &13  & 11 & 11 \\
				\hline
				Rangun & 1 & 1 & 1 & 2 & 13 & 23 & 25 & 24 & 20 & 11 & 4 & 1 \\
				\hline
			\end{tabular}
		}
		
	\vspace{1cm}
	
	\begin{center}
		\begin{tikzpicture}
			\begin{axis}[
				width=\textwidth,
				height=8cm,
				ybar=0.4,
				bar width=8pt,
				enlarge x limits=0.1,
				ylabel={Regentage},
				xlabel={Monat},
				xtick=data,
				xticklabels={Jan,Feb,März,Apr,Mai,Juni,Juli,Aug,Sept,Okt,Nov,Dez},
				legend style={at={(0.5,-0.15)}, anchor=north, legend columns=-1}
				]
				\addplot[fill=blue!70!black] coordinates {(1,10) (2,8) (3,11) (4,8) (5,9) (6,9) (7,11) (8,11) (9,10) (10,13) (11,11) (12,11)};
				\addplot[fill=red!70!black]  coordinates {(1,1) (2,1) (3,1) (4,2) (5,13) (6,23) (7,25) (8,24) (9,20) (10,11) (11,4) (12,1)};
				\legend{\color{black}Amsterdam, \color{black}Rangun}
			\end{axis}
		\end{tikzpicture}
	\end{center}
	
		\subsection{Aufgabe B}
			\noindent
			\begin{minipage}[t]{0.48\textwidth}
				\subsubsection*{Amsterdam}
				\vspace{-3mm}
				{
					\begin{align*}
						\overline{x}_A &= \tfrac{10 + 8 + 11 + 8 + 9 + 9 + 11 + 11 + 10 + 13 + 11 + 11}{12} \\
						&= \boxed{\tfrac{122}{12} \approx 10.17}
					\end{align*}
				}
				{\footnotesize % unwichtige Zwischenschritte kleiner
					\begin{align*}
						s_A^2 &= \frac{1}{12}\sum_{i=1}^{12}(x_i-\overline{x}_A)^2 \\
						&= \frac{(10-10.17)^2 + (8-10.17)^2 + \dots + (11-10.17)^2}{12} \\
						&= \frac{24.67}{12} \approx \boxed{2.06}
					\end{align*}
				}% Ende footnotesize
				\[
				s_A = \sqrt{s_A^2} = \boxed{\sqrt{2.06}\approx 1.43}
				\]
			\end{minipage}\hfill
			\begin{minipage}[t]{0.48\textwidth}
				\subsubsection*{Rangun}
				\vspace{-3mm}
				{\footnotesize
				\begin{align*}
					\overline{x}_R &= \frac{1 + 1 + 1 + 2 + 13 + 23 + 25 + 24 + 20 + 11 + 4 + 1}{12} \\
					&= \boxed{\tfrac{126}{12} = 10.5}
				\end{align*}
				}
				{\footnotesize
					\begin{align*}
						s_R^2 &= \frac{1}{12}\sum_{i=1}^{12}(x_i-\overline{x}_R)^2 \\
						&= \frac{(1-10.5)^2 + (1-10.5)^2 + \dots + (1-10.5)^2}{12} \\
						&= \frac{971}{12} \approx \boxed{80.92}
					\end{align*}
				}
				\[
				s_R = \sqrt{s_R^2} = \boxed{\sqrt{80.92}\approx 8.99}
				\]
			\end{minipage}
	\end{loesung}
	
	\newpage
	
	\begin{loesung}{Aufgabe 4}
		\subsection*{1. Achsenbeschriftung}
		\begin{itemize}
			\item X-Achse: „Merkmal“ bzw. „Anzahl Ereignisse“
			\item Y-Achse: „Relative Häufigkeit [\%]“
			\item Skalierung: $0 \%$ bis $40 \%$ in $10 \%$-Schritten
		\end{itemize}
		
		\subsection*{2. Mittelwert und Standardabweichung}
		
		Gegeben:
		\[
		x_i = 0,1,2,3, \qquad f_i = 0.1,0.2,0.3,0.4
		\]
		
		\paragraph{Mittelwert:}
		\[
		\overline{x} = \sum_i x_i \cdot f_i 
		= 0\cdot 0.1 + 1\cdot 0.2 + 2\cdot 0.3 + 3\cdot 0.4
		= 2.0
		\]
		
		\paragraph{Varianz:}
		\[
		s^2 = \sum_i f_i \cdot (x_i - \overline{x})^2
		= 0.1\cdot (0-2)^2 + 0.2\cdot (1-2)^2 + 0.3\cdot (2-2)^2 + 0.4\cdot (3-2)^2
		= 1.0
		\]
		
		\paragraph{Standardabweichung:}
		\[
		s = \sqrt{1.0} = 1.0
		\]
		
		\subsection*{3. Gleich hohe Säulen}
		
		Falls alle $f_i = 0.25$ gilt:
		
		\paragraph{Mittelwert:}
		\[
		\overline{x} = 0\cdot0.25 + 1\cdot0.25 + 2\cdot0.25 + 3\cdot0.25 = 1.5
		\]
		
		\paragraph{Varianz:}
		\[
		s^2 = 0.25\cdot(0-1.5)^2 + 0.25\cdot(1-1.5)^2 
		+ 0.25\cdot(2-1.5)^2 + 0.25\cdot(3-1.5)^2 = 1.25
		\]
		
		\paragraph{Standardabweichung:}
		\[
		s = \sqrt{1.25} \approx 1.118
		\]
	\end{loesung}
	
	
	\newpage

	
	\section{Berechnung von Wahrscheinlichkeiten}
	
	\section{Wahrscheinlichkeitsverteilung}
	
	\section{Hypothesentest (für binominalverteilte Zufallsgrößen)}
	
	\section{Prognose- und Konfidenzintervalle (für binomialverteilte Zufallsgrößen)}
	
	\newpage
	
	
	\begin{theo}{Quadratische Ergänzung}
		Sei \(a,b\in \mathbb{R}\). Dann gilt
		\[
		(a+b)^2 = a^2 + 2ab + b^2.
		\]
	\end{theo}
	
	\begin{exem}{Numerisches Beispiel}
		Für \(a=2\), \(b=3\) erhalten wir
		\[
		(2+3)^2 = 2^2 + 2\cdot 2\cdot 3 + 3^2 = 25.
		\]
	\end{exem}
	
	\begin{aufgabe}{Binomische Formel}
		Beweise die zweite binomische Formel: \((a-b)^2 = a^2 - 2ab + b^2\).
	\end{aufgabe}
	
	\begin{loesung}{Lösungsskizze}
		Ausmultiplizieren liefert
		\[
		(a-b)^2 = a^2 - 2ab + b^2.
		\]
	\end{loesung}
	
	\begin{infobox}
		Diese Box ist ein Beispiel für Hinweise, farblich und formal abgesetzt.
	\end{infobox}
	
\end{document}
