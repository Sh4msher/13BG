% --- Sprachen und Kodierung ---
\usepackage[ngerman]{babel}   % deutsche Sprachunterstützung
\usepackage{csquotes}         % korrekte Anführungszeichen

% --- Mathematik ---
\usepackage{amsmath}          % grundlegende Mathematikumgebung
\usepackage{mathtools}        % Erweiterungen für amsmath
\usepackage{physics}          % nützliche Makros für Physik
\usepackage{dsfont}           % Mengensymbole (z. B. \mathds{R})

% --- Schriften ---
% für XeLaTeX/LuaLaTeX: fontspec, unicode-math und Libertinus
\usepackage{fontspec}
\usepackage{unicode-math}
\setmainfont{Libertinus Serif}
\setsansfont{Libertinus Sans}
\setmonofont{Libertinus Mono}
\setmathfont{Libertinus Math}

% --- Layout und Typografie ---
\usepackage[top=3cm, bottom=3cm, left=2.5cm, right=2.5cm]{geometry}
\usepackage{microtype}        % schönerer Randausgleich
\usepackage[onehalfspacing]{setspace} % 1,5-facher Zeilenabstand
\usepackage{tocloft}          % Inhaltsverzeichnis anpassen
\renewcommand{\cftsecleader}{\cftdotfill{\cftdotsep}} 

% --- Farben und Boxen ---
\usepackage{xcolor}
\newcommand{\ricardo}[1]{%
	\colorbox{ForestGreen}{\color{white}\textsf{\textbf{Ricardo}}}%
	\textcolor{ForestGreen}{#1}%
}

% --- Tabellen und Listen ---
\usepackage{tabularx}
\usepackage{longtable}
\usepackage{dcolumn}
\usepackage{adjustbox}

% --- Grafiken ---
\usepackage{graphicx}
\usepackage{here}
\usepackage{floatflt}     % Bilder im Fließtext (eher alt)
\usepackage{epsfig}       % alte EPS-Unterstützung
\usepackage{epstopdf}     % Konvertierung EPS -> PDF

% --- Zitate und Literatur ---
\usepackage{cite}
\usepackage{bibgerm}      % deutsche BibTeX-Stile (alt; besser biblatex)

% --- Sonstiges ---
\usepackage{acro}         % Abkürzungsverzeichnis
\usepackage{blindtext}
\usepackage{lipsum}
\usepackage{listings}     % Quellcode
\usepackage{lettrine}     % Initialen
\usepackage[cute inductors,siunitx]{circuitikz} % Schaltpläne
