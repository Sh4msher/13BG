% ============== Dark-Mode Skript (Deutsch-Grundkurs, literarischer Stil) ==============
\documentclass[11pt,a4paper,oneside]{article}

% -------------------- Engine & Fonts --------------------
\usepackage{fontspec}
\defaultfontfeatures{Ligatures=TeX,Scale=MatchLowercase}
\setmainfont{Inter}
%\setmainfont{Roboto}

%\setmainfont{EB Garamond}        % Fließtext (ersetzen, falls nicht installiert)
%\setsansfont{Source Sans Pro}   % Überschriften
%\usepackage{unicode-math}
%\setmathfont{Libertinus Math}   % für Formeln

%\setmainfont{TeX Gyre Pagella}
%\setsansfont{Source Sans Pro}



% -------------------- Pakete --------------------
\usepackage{acro}
\usepackage[ngerman]{babel}
\usepackage{microtype}
\usepackage{geometry}
\usepackage{titlesec}
\usepackage{fancyhdr}
\usepackage{xcolor}
\usepackage{pagecolor}
\usepackage{tikz}
\usetikzlibrary{calc}
\usepackage[most]{tcolorbox}
\tcbuselibrary{skins,breakable,theorems}
\usepackage{enumitem}
\usepackage{caption}
\usepackage{graphicx}
\usepackage{hyperref}

% -------------------- Layout --------------------
\geometry{
	left=28mm, right=28mm, top=28mm, bottom=28mm,
	marginparwidth=36mm, marginparsep=6mm
}
\setlist{nosep}

% -------------------- Farben (wärmere, harmonisierte Darkmode-Palette) --------------------
\definecolor{PageBG}{RGB}{34,30,28}        % leicht wärmeres Anthrazit (nicht zu hart)
\definecolor{TextCream}{RGB}{245,240,232}  % weiches Cremeweiß (Text)
\definecolor{AccentRed}{RGB}{130,38,48}    % Bordeaux für wichtige Akzente / Titel
\definecolor{AccentOcker}{RGB}{180,120,40} % Ocker/Gold als Sekundär-Akzent
\definecolor{AccentGray}{RGB}{150,150,150} % warmes Grau
\definecolor{BoxInner}{RGB}{46,40,36}      % Box-Hintergrund (heller als PageBG)
\definecolor{BoxTint}{RGB}{58,50,46}       % leichte Innen-Tönung für Notizen
\definecolor{MarginalGray}{RGB}{130,130,130}

% Farbabstufung von Rot nach Gelb (für die einzelnen Boxen)
\definecolor{AccentA}{RGB}{130,38,48}   % kräftiges Bordeaux (zitat)
\definecolor{AccentB}{RGB}{170,78,45}   % rot-orangig (interpret)
\definecolor{AccentC}{RGB}{200,110,52}  % orange (hinweis)
\definecolor{AccentD}{RGB}{220,140,55}  % ocker / amber (aufgabe)
\definecolor{AccentE}{RGB}{240,180,70}  % warmes Gelb (lösung)

%\pagecolor{PageBG}
%\color{TextCream}

% -------------------- Kopf / Fuß --------------------
\pagestyle{fancy}
\fancyhf{}
\renewcommand{\headrulewidth}{0pt}
\setlength{\headheight}{14pt}
\fancyfoot[C]{%
	\scriptsize\scshape Ferdinand-Braun Schule \;•\;  Deutsch Grundkurs Q3 \;•\; \thepage
	% dekorative Linie (wird in der Fußzeile für jede Seite gezeichnet)
	\begin{tikzpicture}[remember picture,overlay]
		\draw[line width=0.8pt,color=AccentRed!70!black]
		($(current page.south west)+(28mm,22mm)$) -- ($(current page.south east)+(-28mm,22mm)$);
	\end{tikzpicture}%
}

% Section (größte Überschrift) → tiefrot (AccentA)
\titleformat{\section}
{\normalfont\Large\bfseries\sffamily\color{AccentA}}
{\thesection}{1em}{}

% Subsection → rot-orangig (AccentB)
\titleformat{\subsection}
{\normalfont\large\bfseries\sffamily\color{AccentC}}
{\thesubsection}{0.8em}{}

% Subsubsection → orange (AccentC)
\titleformat{\subsubsection}
{\normalfont\normalsize\itshape\sffamily\color{AccentE}}
{\thesubsubsection}{0.6em}{}

% -------------------- TCBOX Grundstil (sanfter, lesefreundlich) --------------------
\tcbset{
	mybase/.style={
		enhanced,
		breakable,
		boxrule=0.9pt,                 % Randstärke (wird pro Box farbig)
		%colframe=AccentA,              % Default-Frame (wird in Box-Definitionen überschrieben)
		%colback=PageBG,                % "transparent" => gleiche Farbe wie Seite
		%colupper=TextCream,            % Textfarbe innerhalb der Box
		arc=4mm,
		boxsep=6pt,
		left=12pt,right=12pt,top=12pt,bottom=12pt,
		before skip=10pt, after skip=10pt,
		fonttitle=\sffamily\bfseries\small,
		title after break=\vspace{4pt},
		boxed title style={
			arc=3mm,
			boxrule=0pt,
			left=6pt,right=6pt,top=2pt,bottom=2pt
		}
	}
}

% -------------------- Box-Typen (jeweils eigene Rahmenfarbe + Titelsticker) --------------------
% ZITAT (tiefrot)
\newtcolorbox{zitat}[1][]{%
	mybase,
	colframe=AccentA,
	%colback=PageBG,                % transparent
	colbacktitle=AccentA,          % Titelsticker hat die Rahmenfarbe
	coltitle=TextCream,
	fontupper=\itshape,
	attach boxed title to top left={yshift=-2mm, xshift=8mm},
	title={Zitat},
	#1
}

% INTERPRETATION (rot->orange)
\newtcolorbox{interpret}[1][]{%
	mybase,
	colframe=AccentB,
	%colback=PageBG,
	colbacktitle=AccentB,
	coltitle=TextCream,
	attach boxed title to top left={yshift=-2mm, xshift=8mm},
	title={Interpretation},
	#1
}

% HINWEIS (orange)
\newtcolorbox{hinweis}[1][]{%
	mybase,
	colframe=AccentC,
	%colback=PageBG,
	colbacktitle=AccentC,
	coltitle=TextCream,
	attach boxed title to top left={yshift=-2mm, xshift=8mm},
	title={Hinweis},
	#1
}

% AUFGABE (amber / ocker)
\newtcolorbox[auto counter,number within=section]{aufgabe}[2][]{%
	mybase,
	colframe=AccentD,
	%colback=PageBG,
	colbacktitle=AccentD,
	coltitle=TextCream,
	attach boxed title to top left={yshift=-2mm, xshift=8mm},
	title={Aufgabe~\thetcbcounter: #2},
	#1
}

% LÖSUNG (gelb)
\newtcolorbox[use counter from=aufgabe]{loesung}[2][]{%
	mybase,
	colframe=AccentE,
	%colback=PageBG,
	colbacktitle=AccentE,
	coltitle=TextCream,
	attach boxed title to top left={yshift=-2mm, xshift=8mm},
	title={Lösung~\thetcbcounter: #2},
	#1
}


% kleiner Befehl für Datumsanzeige rechts
\newcommand{\lessondate}[1]{\noindent\hfill\textcolor{MarginalGray}{\textsc{#1}}\\\vspace{0.5cm}}

% -------------------- Titelblatt --------------------
\newcommand{\MakeArtTitle}[4]{%
	\begin{titlepage}
		\centering
		\vspace*{24mm}
		{\sffamily\bfseries\huge #1 \par}
		\vspace{8mm}
		{\Large \scshape#2 \par}
		\vspace{16mm}
		{\large\scshape #3 \par}
		\vspace{6mm}
		{\small\color{MarginalGray} #4 \par}
		\vspace{8mm}
		{\small\color{MarginalGray}\today \par}
		\vfill
		%\begin{tikzpicture}
		%	\draw[line width=1.2pt,color=AccentRed!60!black] (-6,0) -- (6,0);
		%	\draw[line width=0.5pt,color=AccentOcker!40!black] (-6,-0.4) -- (6,-0.4);
		%\end{tikzpicture}
		\vspace{6cm}
		\centering
		\includegraphics[width=0.75\textwidth]{image.png} % Logo anpassen / entfernen
	\end{titlepage}
}

% ==================== Dokumentbeginn ====================

\begin{document}
	
	% Titelblatt
	\MakeArtTitle
	{Grundkurs Deutsch — Q3}
	{Zwischen Tradition und Experiment, Krise und Neuanfang — Skript}
	{Shamsher Singh Kalsi}
	{Berufliches Gymnasium — Ferdinand-Braun Schule \\ Kursleiterin: Frau Dagmar Sieverding}
	
	\tableofcontents
	\clearpage
	
	\section{Einleitung}
	\subsection{lol}
	\subsubsection{lel}
	\lessondate{20.08.2025}\\
	Dieses Skript begleitet den Grundkurs Deutsch: Texte, Zitate, Interpretationsansätze und Übungen werden gesammelt. Die Gestaltung ist dunkel und leseschonend — ideal für Beamer-Notizen oder Ausdrucke auf hochwertigem Papier.
	
	
	\newpage
	
	\section{Sprache und Identität – Sprachkrise als Identitätskrise}
	\lessondate{27.08.2025}\\
	
	\begin{hinweis}
		\subsection*{Hugo von Hofmannsthal (1874 - 1929)}
		\subsubsection*{Weltgeheimnis}
		Der tiefe Brunnen weiß es wohl,\\
		Einst waren alle tief und stumm,\\
		Und alle wußten drum.\\
		
		Wie Zauberworte, nachgelallt\\
		Und nicht begriffen in den Grund,\\
		So geht es jetzt von Mund zu Mund.\\
		
		Der tiefe Brunnen weiß es wohl;\\
		In den gebückt, begriffs ein Mann,\\
		Begriff es und verlor es dann.\\
		
		Und redet' irr und sang ein Lied –\\
		Auf dessen dunklen Spiegel bückt\\
		Sich einst ein Kind und wird entrückt.\\
		
		Und wächst und weiß nichts von sich selbst\\
		Und wird ein Weib, das einer liebt\\
		Und – wunderbar wie Liebe gibt!\\
		
		Wie Liebe tiefe Kunde gibt! – \\
		Da wird an Dinge, dumpf geahnt,\\
		In ihren Küssen tief gemahnt...\\
		
		In unsern Worten liegt es drin,\\
		So tritt des Bettlers Fuß den Kies,\\
		Der eines Edelsteins Verlies.\\
		
		Der tiefe Brunnen weiß es wohl, \\
		Einst aber wußten alle drum,\\
		Nun zuckt im Kreis ein Traum herum.\\
		%\footnote{\href{https://www.deutschelyrik.de/weltgeheimnis.html}{Quelle}}
	\end{hinweis}
	
	\newpage
	
	\begin{aufgabe}{Hausaufgabe: Buch Seite 291}
		\begin{itemize}[left=15mm]
			\item [Aufgabe 1:] Stellen Sie die Kennzeichnung von Vergangenheit und Gegenwart gegenüber, wie das lyrische Ich sie vornimmt. 
			\item [Aufgabe 2:] Erklären Sie die Überschrift des Gedichts 
			\item [Aufgabe 3:] Untersuchen Sie die formale und sprachliche Gestaltung des Gedichts und setzen Sie sie in Beziehung zum Inhalt.
		\end{itemize}
	\end{aufgabe}

	
	\begin{loesung}{Aufgabe 1}
	In der Vergangenheit „Einst (aber) wussten alle drum“ wird durch das Adverb „Einst“ ein zeitlich klar abgeschlossener Zustand markiert. Das Präteritum „wussten“ beschreibt ein allgemeines, gesichertes Wissen, das keinen weiteren Kommentar mehr benötigt. Die Gegenwart dagegen zeigt sich im Präsens: „[es] zuckt“. Hier steht „Nun“ im Kontrast zu „Einst“ und verweist auf die aktuelle, nicht abgeschlossene Situation. Während früher ein gemeinsames Wissen selbstverständlich war, bleibt heute nur ein unsicheres „Zucken“ oder ein kreisender Traum übrig.
	\end{loesung}
	
	\begin{loesung}{Aufgabe 2}
		%Die Überschrift ist eine Zusammensetzung (womöglich Neologismus) aus "Welt" und "Geheimnis", welche semantisch ein Mysterium, um das zugrundeliegende Fundament der Co-existenz von Individuen, bildet. Darunter liese sich im allgemeinen Verstehen, dass innerhalb des irdischen Lebens die existenzielle Funktion in frage gestellt wird, wie der Ursprung, das Fortbestehen und die Zukunft aussehen. In dem Kontext des Gedichtes ließe sich die Überschrift aber anders verstehen. Das Gedicht thematisiert die missverständliche reproduktion von deplazierten Inhalten in einem Sachverhalt, sodass der wahre Kern des Sachverhaltes verzerrt oder verloren geht. In diesem Sinne könnte man ebenfalls von dem Weltgeheimnis sprechen, da die Wahrnehmung und Integration in die Persönlichkeitsentwicklung eine hohe Diskrepanz zur Realität und dem Kontext aufweisen.
		Die Überschrift ist eine Zusammensetzung aus „Welt“ und „Geheimnis“ und bezeichnet ein großes, verborgenes Mysterium. Allgemein lässt sich darunter das Rätsel um Ursprung, Fortbestehen und Zukunft des menschlichen Lebens verstehen. Im Kontext des Gedichts erhält der Titel jedoch eine speziellere Bedeutung: Es geht um ein Wissen, das in seiner ursprünglichen Klarheit verloren gegangen ist. Durch die mündliche Weitergabe von „Mund zu Mund“ wird der Kern verzerrt oder unverständlich. So wird das „Weltgeheimnis“ im Gedicht zum Bild für etwas, das zwar in Sprache und Erinnerung weiterlebt, aber nie mehr vollständig erfasst werden kann.  
	\end{loesung}
	
	\begin{loesung}{Aufgabe 3}
	Formal besteht das Gedicht aus acht Strophen mit jeweils drei Versen. Auffällig sind die häufigen Wiederholungen von Sätzen und Satzstrukturen, einige Enjambements sowie polysyndetische Verknüpfungen. Dies erinnert an ein Mantra, das paraphrasierend in tranceartiger Wiederholung gesprochen wird. Die Liedhaftigkeit verstärkt den Eindruck des Phrasenhaften: Inhalte werden knapp weitergegeben und dabei in andere Kontexte verschoben. So stimmen Form und Inhalt überein: Die musikalisch-volkstümliche Gestaltung verdeutlicht, wie ursprüngliches Wissen verflacht und schließlich als kurze, formelhafte Phrasen als vermeintliche Wahrheit fortbesteht.
	\end{loesung}
	
	
	\newpage
	
	
	Rote Dächer u.A. Holz 
	
	\begin{itemize}
		\item Kleinstadtatmosphäre, Idylle, Normalität auch Arbeit 
		\item Aufzähligungen (assoziativ)
		\item kurze Ausrufe 
		\item Partizipen, Adverben, Adjektive 
		\item bildlich, persönlich 
		\item Details, kleinigkeiten 
		\item Sekundenstil 
	\end{itemize}
	
	
	\newpage
	
	\lessondate{09.09.2025}\\
	
	\subsection{Übung Literatur der Jahrhundertwende - Verschränkung der Epochen}
	
	\begin{aufgabe}{Hausaufgabe zum 10.9.25}
	Analyse Gedicht aus Unterricht von Arno Holz (bitte den anderen austeilen, die nicht da waren)
	Findet sich bei Lehrerfortbildung-bw.de unter Lyrik und Jahrhundertwende (Nummer 6)
	\end{aufgabe}
	
	\begin{loesung}{Einleitung und Inhaltsangabe}
		Das Gedicht "In den Grunewald" wurde von Arno Holz (1863-1929) verfasst und erschien 1891. Es ist der Epoche des Naturalismus zuzuordnen. Inhaltlich schildert das Werk einen volksfestartigen Massenausflug aus Berlin in den Grunewald und stellt die Diskrepanz zwischen städtischer Enge und der erhofften Naturidylle dar.
		
		Das Gedicht beschreibt einen Pfingstausflug von Berlinern in den Grunewald. Es beginnt mit der morgendlichen Abreise, als die Stadt Massen von Menschen in Sonderzügen in die Natur entlässt. Die Menschen strömen aus allen Richtungen in Bussen und Zügen herbei, begleitet von Musik und Gesang. Die ausgelassene und laute Stimmung des Tages wird durch die Nennung von Liedtiteln wie "Pankow, Pankow, Pankow, Kille, Kille" und "Holzauktion" eingefangen. Der zweite Teil des Gedichts wechselt abrupt zur Nacht. Die anfängliche Freude weicht einer melancholischen, fast trostlosen Atmosphäre. Nur noch der Lärm eines Leierkastens ist zu hören, und eine brennende Zigarre sowie ein Pfingstkleid verschwinden in der Dunkelheit, was auf heimliche Rendezvous hindeutet. Am Ende steht das ironische Bild der Mondgöttin Luna, die über die menschliche Suche nach der "blauen Blume" inmitten von Müll und Abfall lächelt.
		
		Das Gedicht besteht aus freien Versen ohne festes Reimschema oder Metrum. Es hat keine klassische Strophenform. Die unregelmäßigen Zeilenlängen und der fragmentarische Satzbau sind typisch für den Naturalismus. Das Gedicht ist in zwei Teile gegliedert, die durch einen Zeitsprung voneinander getrennt sind.
	\end{loesung}
	
	\newpage
	
	\begin{loesung}{Analyse}
		Arno Holz’ Gedicht \textit{In den Grunewald} (Autor: Arno Holz, 1863–1929; Drucklegung: 1891) schildert in konzentrierten Impressionen einen massenhaften Ausflug der Berliner Bevölkerung in den Grunewald und verknüpft diese Alltagsszene mit einer pointierten kulturkritischen Perspektive. In knappen Szenenwechseln führt der Text von der frühen Morgenszene, in der „Berlin seine Extrazüge“ aussendet, über die lauten, städtischen Eindrücke während der Anreise bis hin zu einer nächtlichen Auflösung der Feierlichkeit in Abfall, Müdigkeit und ironischer Distanz. Formal arbeitet das Gedicht mit freien Versen und fragmentarischer Satzführung; sein Aufbau ist nicht strophisch-geordnet, sondern eher assoziativ und montagehaft, was die Beobachterhaltung und die dokumentarische Absicht des Autors unterstützt.
		
		Inhaltlich beschreibt der erste Abschnitt die Anreise: Ortsnamen und Hörerlebnisse treten als konkrete, unvermittelte Sinneseindrücke auf („Pankow, Pankow, Pankow, Kille, Kille“ usw.), die den Eindruck einer überfüllten, lauten und instrumentierten Volksbewegung vermitteln. Diese direkte Wiedergabe von äußeren Reizen — kurze Ausrufe, Titelnennungen, Aufzählungen — entspricht der naturalistischen Absicht, Wirklichkeit in sekundenschnellen Eindrücken zu rekonstruieren (Sekundenstil). Die Sprache ist dabei bewusst nüchtern und dokumentarisch; Bewertendes wird auf der sprachlichen Ebene weitgehend vermieden, sodass die Szenerie für sich sprechen soll.
		
		Der zweite Abschnitt vollzieht einen Stimmungsumschlag: Aus dem tagsüber lauten, bunten Treiben wird Nacht. Die Festlichkeit erschöpft sich, Reste menschlicher Präsenz und Relikte des Tages (ein Pfingstkleid, eine brennende Zigarre) verschwinden „hinter den Bahndamm“; der Leierkasten quäkt noch — ein letztes Echo der Volksmusik. In dieser Dämmerung tritt die Metaphorik deutlicher hervor: Die „blaue Blume“ — ein tradiertes Symbol der Romantik für Sehnsucht und Transzendenz — wird nicht in einer erhabenen Natur gefunden, sondern inmitten von „weggeworfenem Stullenpapier und Eierschalen“ gesucht. Das Bild wirkt desillusionierend: Romantische Sehnsucht prallt mit profaner Alltagswirklichkeit zusammen und verliert ihren sakralen Glanz.
		
		Auf interpretativer Ebene lässt sich das Gedicht als kritische Sozial- und Kulturbeobachtung lesen. Holz dokumentiert nicht nur ein Volksfest, sondern setzt die populäre Freizeitkultur in Beziehung zu städtischer Massenbewegung, Kommerzialisierung und Entzauberung. Die technischen und urbanen Elemente (Extrazüge, Kremser, Turnerzüge) sowie die akustischen Bruchstücke erzeugen einen Eindruck von Organisiertheit und Konsum, während der nächtliche Restmüll und das ironische Lächeln der Luna eine distanzierte Reflexionsebene eröffnen: Die Sehnsucht nach dem „Anderen“ bleibt oberflächlich, wird durch das Gedränge, durch Abfall und banale Reste relativiert. Die Figur der Luna fungiert dabei als beobachtende, fast mythologische Instanz, die mit einem milden, distanzierten Lächeln über die menschlichen Bemühungen wacht — ein Moment, das sowohl Ironie als auch melancholische Bemerkung enthält.
		
		Sprachlich und stilistisch ist das Gedicht von Minimalismus und Lauthaftigkeit geprägt: kurze Wortfolgen, Inkongruenzen und Zitatwiedergaben erzeugen Authentizität; die fehlende metrische Gliederung unterstreicht die dokumentarische Tonalität. Die beiden Zeitebenen (Morgen — Nacht) strukturieren die Erzählung temporal und thematisch und ermöglichen den Kontrast zwischen Inszenierung und Ausklang. Als Gattungsbezug ist Naturalismus als prägende Strömung plausibel: die genaue Beobachtung sozialer Praxis, die Vermeidung idealisierender Verklärung und die protokollarische Wiedergabe von Alltagsmaterial sind typische Merkmale.
		
		Abschließend lässt sich festhalten: \textit{In den Grunewald} verbindet im knappen, sinnlich-dokumentarischen Stil unmittelbare Szenenbeschreibung mit einer subtilen, ironisch-melancholischen Kritik an Bürgerlichkeit und Romantisierung der Massenkultur. Das Gedicht lädt dazu ein, die Momentaufnahmen nicht nur als Schilderung, sondern als kulturelle Diagnose zu lesen — als Porträt einer Epoche, die an der Schwelle von Natursehnsucht, Massentourismus und urbaner Entzauberung steht.
	\end{loesung}
	
	\newpage
	
	
	\section{Subjektivität und Verantwortung - anthropologische Grundfragen}
	\section{Epochenumbruch 19./20. Jahrhundert – literarische Moderne im frühen 20. Jahrhundert}
	\section{Neuanfänge nach historischen Zäsuren 1945/1990}
	\section{Film und Literatur}
	
	\newpage
	
	\section{Beispieltexte und Analyse}
	
	\begin{zitat}
		Es ist alles eitel. \\
		— Andreas Gryphius
	\end{zitat}
	
	\begin{interpret}
		Das Gedicht arbeitet mit barocker Vanitas-Symbolik. Die Sprache ist pointiert, Antithesen und Metaphern strukturieren das Bild.
	\end{interpret}
	
	\begin{hinweis}
		Achte bei der Analyse auf rhetorische Mittel (Alliteration, Antithese, Hyperbel) und auf das historische Setting (Barock).
	\end{hinweis}
	
	\begin{aufgabe}{Kurze Analyse}
		Analysiere das obige Zitat bezüglich Bildsprache und Intention in 6–8 Sätzen.
	\end{aufgabe}
	
	\begin{loesung}{Hinweise}
		Mögliche Punkte: Vanitas-Motiv, Gegenüberstellung von Eitelkeit vs. Ewigkeit, Verwendung von starken Bildwörtern. (Stichpunkte genügen.)
	\end{loesung}
	
\end{document}
