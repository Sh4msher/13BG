% ============== Adjusted LaTeX Template (XeLaTeX / LuaLaTeX) ==============
\documentclass[11pt,a4paper,oneside]{article}

% -------------------- Engine & Fonts --------------------
\usepackage{fontspec}
\setmainfont{Libertinus Serif}
\setsansfont{Libertinus Sans}
\usepackage{unicode-math}
\setmathfont{Libertinus Math}

% -------------------- Packages --------------------
\usepackage{paralist}
\usepackage{tasks}
\usepackage{acro}
\usepackage{microtype}
\usepackage{geometry}
\usepackage{titlesec}
\usepackage{fancyhdr}
\usepackage{xcolor}
\usepackage{pagecolor}
\usepackage{tikz}
\usetikzlibrary{shadows,calc}
\usepackage[most]{tcolorbox}
\tcbuselibrary{skins,breakable,theorems}
\usepackage{enumitem}
\usepackage{caption}
\usepackage{everypage}
\usepackage{amsmath}
\usepackage{graphicx}

% -------------------- Layout --------------------
\geometry{
	left=28mm, right=28mm, top=28mm, bottom=28mm,
	marginparwidth=36mm, marginparsep=6mm
}

% -------------------- Colors (adjusted) --------------------
% Page background: slightly lighter yellow (helleres gelb)
\definecolor{PageBG}{RGB}{255,253,230} % very pale warm yellow
\definecolor{dings}{RGB}{255,255,240}
\definecolor{TextBlack}{RGB}{34,34,34}

% Accent palette (flow: orange -> red -> black)
\definecolor{AccentOrange}{RGB}{255,140,0}   % strong orange
\definecolor{AccentRed}{RGB}{220,20,60}      % crimson-ish red
\definecolor{AccentDark}{RGB}{20,20,20}      % near-black for final boxes

% Subtle box backgrounds (tints derived from accents)
\colorlet{BoxOrange}{AccentOrange!8!white}
\colorlet{BoxRed}{AccentRed!6!white}
\colorlet{BoxDark}{AccentDark!6!white}
\definecolor{MarginalGray}{RGB}{112,128,144}


\definecolor{TextCream}{RGB}{246,241,234}       % creme / sehr hell

\definecolor{AccentBlue}{RGB}{76,104,150}       % akademisches Blau (unverändert)
\definecolor{AccentViolet}{RGB}{105,77,145}     % seriöses Violett (unverändert)

% Hint: mehr ins violett/pinke ziehen (Hauch von Fuchsia)
\definecolor{AccentMagenta}{RGB}{140,60,120}       % violett-pink (sanfter Übergang)

% Magenta: rötlicher, dichter, aber nicht grell
\definecolor{AccentHint}{RGB}{170,40,60}     % rötliches Magenta (mehr Rotanteil)

% Elegantes Rot: orientiert sich am bisherigen "AccentHint" (tiefes Weinrot)
\definecolor{AccentRed}{RGB}{120,10,25}         % elegantes Weinrot / tiefes Rot

\definecolor{BoxBackground}{RGB}{15,9,27}       % Box-Hintergrund
\definecolor{MarginalGray}{RGB}{150,150,160}    % Rand-Datum




	
\pagecolor{dings}
\color{TextBlack}

% -------------------- Header / Footer --------------------
\pagestyle{fancy}
\fancyhf{}
\renewcommand{\headrulewidth}{0pt}
\setlength{\headheight}{14pt}
\fancyfoot[C]{%
	\vspace{6pt}%
	{\scriptsize\scshape\color{MarginalGray}Ferdinand-Braun Schule \quad • \quad Advanced English Course (Q3) \quad • \quad \thepage}%
	\begin{tikzpicture}[remember picture,overlay]
		\draw[line width=0.9pt,color=AccentOrange!70!black] ($(current page.south west)+(28mm,22mm)$) -- ($(current page.south east)+(-28mm,22mm)$);
	\end{tikzpicture}%
}

% -------------------- Title Styles (follow color flow) --------------------
\titleformat{\section}
{\normalfont\large\bfseries\color{AccentOrange!95!black}}
{\thesection}{1em}{}
\titleformat{\subsection}{\normalfont\normalsize\bfseries\color{AccentRed!85!black}}{\thesubsection}{0.8em}{}
\titleformat{\subsubsection}{\normalfont\normalsize\bfseries\color{AccentDark!85!black}}{\thesubsubsection}{0.6em}{}

% ========================= TCBOX BASE STYLE (sauber, nicht ineinander verwoben) =========================
\tcbset{
	mybase/.style={
		enhanced,
		breakable,
		boxrule=0.8pt,
		colframe=black!20,
		colback=white,
		colupper=TextBlack,
		arc=3mm,
		boxsep=6pt,
		left=14pt,right=14pt,top=12pt,bottom=12pt,
		before skip=12pt, after skip=12pt, % mehr Abstand, verhindert "ineinander verwoben"
		boxed title style={
			arc=4mm,
			boxrule=0pt,
			interior style={left color=white,right color=white},
			attach boxed title to top left={yshift=-\tcboxedtitleheight/2,xshift=8mm}
		},
		fonttitle=\sffamily\bfseries\small,
		title after break=\vspace{6pt}
	}
}

% -------------------- BOX TYPES (klarer, sequentieller Farbübergang) --------------------
\newtcolorbox[auto counter,number within=section]{theorem}[2][]{%
	mybase,
	colframe = AccentOrange!85!black,
	colback = BoxOrange,
	colbacktitle = AccentOrange!85!black,
	coltitle = PageBG,
	title = {Theorem~\thetcbcounter: #2},
	#1
}

\newtcolorbox[auto counter,number within=section]{example}[2][]{%
	mybase,
	colframe = AccentRed!85!black,
	colback = BoxRed,
	colbacktitle = AccentRed!85!black,
	coltitle = PageBG,
	title = {Example~\thetcbcounter: #2},
	#1
}

\newtcolorbox[auto counter,number within=section]{task}[2][]{%
	mybase,
	colframe = AccentDark!80!black,
	colback = BoxDark,
	colbacktitle = AccentDark!90!black,
	coltitle = PageBG,
	title = {Task~\thetcbcounter: #2},
	#1
}

\newtcolorbox[use counter from=task]{solution}[2][]{%
	mybase,
	colframe = AccentDark!80!black,
	colback = BoxDark,
	colbacktitle = AccentDark!90!black,
	coltitle = PageBG,
	title = {Solution~\thetcbcounter: #2},
	#1
}

% -------------------- Note Box (neutral, but follows dark accent) --------------------
\newtcolorbox{infobox}[1][]{%
	mybase,
	colframe = AccentDark!60!black,
	colback = BoxDark,
	colbacktitle = AccentDark!80!black,
	coltitle = PageBG,
	title = {Note},
	#1
}

\newcommand{\lessondate}[1]{
	\noindent\hfill\textcolor{gray}{\textsc{#1}} \\
	\vspace{0.5cm}
}



	
% ==================== Feines Titelblatt ====================
\newcommand{\MakeArtTitle}[4]{%
	\begin{titlepage}
		\vspace*{18mm}
		\begin{center}
			\vspace{12mm}
			{\huge\color{AccentOrange!90!black} #1 \par}
			\vspace{6mm}
			{\Large\itshape\color{AccentRed!70!black} #2 \par}
			\vspace{10mm}
			{\Large\scshape\color{AccentDark} #3 \par}
			\vspace{6mm}
			{\small\color{MarginalGray} #4 \par}
			\vspace{5mm}
			{\small\color{MarginalGray} \today \par}
		\end{center}
		\vspace{7.5cm}
		\centering
		\includegraphics[width=0.75\textwidth]{image.png} % Logo einfügen (Pfad anpassen)
	\end{titlepage}
}


% -------------------- Start Document --------------------
\begin{document}

	

	
		
	% Titelblatt
	\MakeArtTitle{
		Grundkurs Englisch Q3 Hessen}
	{Skript}
	{Shamsher Singh Kalsi}
	{Berufliches Gymnasium — Ferdinand-Braun Schule \\ Kursleiter: Herr Klaus Heinz}
	
	\tableofcontents
	\bigskip
	\clearpage
	
	
	
	\section{Einleitung}
	Für dieses Halbjahr wird das Buch The Circle von Dave Eggers behandelt. Nach den Herbstferien wird das Buch bearbeitet.
	
	\section{Human dilemmas in fiction and real life}
	
	
	
	
	\section{Modelling the future}
	\section{Gender issues}
	
	\lessondate{25.05.2025}\\
	
	Workbook Page 32. 
	
	\subsection{Diversity in the Workplace}
	\begin{itemize}
		\item Diversity means that a company is made up of a wide range of individuals - gender, race, age, educational background.
		\item Issues; Takes a lot of work to take care of the perspective of the others. maybe people in a team from the same background work more efficient 
		\item divers teams can be more creative and therefore more efficient to solve problems 
		\item downsides: increase the risk of disagreemends and delay projects
		\item cultural concepts: when people are not open enough to accept others opinions
		\item can be successful if everyone shows commitment and is accepting one another
		\item 
	\end{itemize}
	
	\subsubsection{Cartoon Analysis}
	\lessondate{01.09.2025}\\
	The present cartoon is made by Loren Fishman and is about discrimination and the lack of diversity in workplaces. 
	
	You can see a generic workplace with several men in the center of the cartoon. There are three pairs of men talking to each other -  one on the middle left, one on the middle right and one in the left front. Three men isolated and work by them self. All men look similiar with a black suit, a white shirt, a black tie and a slightly colored skin with the same haircut. All do share the same blank facial expression. 
	On the bottom of the cartoon is a quote "Oh, you will love it here. Nobody treats you any differently just because of your age, race or gender". 
	The first association is ofcourse the paradox that on one hand the cartoon is pointing out tolerance in the company while on the other hand everybody is the same. 
	
	
	\vspace{20mm}
	The cartoon is striking because every person looks completely interchangeable, as if individuality has been erased. Each character seems to be a product of the capitalist machine, designed not to think but to execute. Diversity, difference, and true personality have no place in such a world, because the system functions best when people can be treated identically, without responsibility or deeper recognition. The blank facial expressions highlight this dehumanization: workers are reduced to machines, stripped of identity. The cartoon cleverly exposes how capitalism prefers uniformity over individuality, efficiency over humanity, and empty words of equality over genuine diversity.
	
	\newpage
	
	\begin{task}{\lessondate{04.09.2025}}
		\begin{itemize}
			\item AB diversity in Gender Part 2 
			\item Book Page 36 No.6 $\rightarrow$ letter to the editor (Page 210)
			\item AB - part 3 grammar
		\end{itemize}
	\end{task}
	
	\begin{solution}{PART 2}
		\begin{itemize}
			\item Mother (Actions)
			\begin{itemize}
				\item 
			\end{itemize}
			\item Santi (Appearance / Actions)
			\begin{itemize}
				\item 
			\end{itemize}
			\item Brother (Actions)
			\begin{itemize}
				\item 
			\end{itemize}
		\end{itemize}
	\end{solution}
	
	
	\newpage
	
	\section*{Santi Ceballos}
	
	\subsection*{Appearance}
	\begin{itemize}
		\item Wavy, golden-brown hair
		\item Past their shoulders
		\item Thick bangs, long eyelashes
		\item Oversized aviator glasses
		\item Round face, clear skin (due to hormone blockers)
		\item Wears oversized windbreaker and skinny jeans
	\end{itemize}
	
	\subsection*{Actions}
	\begin{itemize}
		\item Worked to change Arizona curriculum laws
		\item Made an educational video about being gender nonconforming
		\item Attended Camp Born This Way
		\item Came out as non-binary and chose they/them pronouns
		\item Transferred to a progressive charter school
		\item Joined a lawsuit to repeal \emph{No Promo Homo} law
	\end{itemize}
	
	
	\section*{Mother: Carol Brochin}
	
	\subsection*{Actions}
	\begin{itemize}
		\item Education professor
		\item Teaches empathy and understanding for LGBTQ+ students
		\item Enrolled Santi in Camp Born This Way
		\item Member of camp steering committee
		\item Found counseling for Santi
	\end{itemize}
	
	
	\section*{Brother: Joaquin Ceballos}
	
	\subsection*{Actions}
	\begin{itemize}
		\item Changed schools together with Santi
		\item Protected Santi from bullying and teasing
		\item Left old school because of too much ``busy work''
	\end{itemize}
	
	\newpage
	
	At school, several things happened to Santi that shaped their experience and identity. Other kids sometimes called Santi "it," treating them as if they were a different species. At their old public school, Santi was teased for playing with the girls while still using the boys’ bathroom, which made them feel isolated. Later, in seventh grade at their new charter school, they initially felt more comfortable because the environment was more progressive, with a gender-neutral restroom and a focus on social justice. But when it came to sex education, the boys and girls were separated, and Santi was forced to choose a side. The curriculum, based only on heterosexuality, excluded them completely. This situation triggered dysmorphia and suicidal thoughts, showing how hostile or inadequate school structures could be for a non-binary student.
	
	When Santi’s mother, Carol Brochin, says that Santi "has had a long and sometimes painful journey," she refers to the ongoing struggle her child has faced with identity and recognition. The journey has been "long" because it involves years of negotiating between who Santi knows themselves to be and what schools, peers, and laws expect them to be. It has been "painful" because Santi endured bullying, misgendering, exclusion from sex education, and the psychological toll of not being seen or respected. At the same time, there were moments of empowerment, like at Camp Born This Way or at the new school, but even those gains often came with new vulnerabilities. Brochin’s statement captures both the suffering and resilience in Santi’s process of becoming visible as a non-binary teen.
	
	\subsection*{PART 3}
	Sir,
	
	The story of Santi Ceballos and their family struck me deeply. In a society where non-binary teenagers are often met with ridicule or silence, Santi’s family reacted with openness, solidarity, and concrete action. Their mother, Carol Brochin, did not shy away from her child’s struggle but actively sought support systems, such as enrolling Santi in Camp Born This Way and later finding professional counseling when the pressure became overwhelming. Rather than ignoring or suppressing the issue, she transformed her role as an education professor into one of advocacy, reminding future teachers that empathy is not an optional virtue but a professional necessity.
	
	Equally important is the role of Santi’s brother, Joaquin. He could easily have distanced himself, as many siblings do when one child becomes the center of attention. Instead, he changed schools with Santi and took it upon himself to protect them from mockery. His quiet gesture of solidarity is a reminder that acceptance does not always come in the form of slogans or activism, but also through small, consistent acts of care.
	
	This family’s reaction highlights what is too often missing in public discourse: a combination of love and courage. Love alone is not enough, if it remains private; courage alone can be brittle if it lacks compassion. The Ceballos family has shown both, and their example deserves attention in a climate where non-binary youths are still fighting for the simple right to be recognized.
	
	Yours faithfully,
	\newpage
	
	\subsection{Ehtnicity}
	\lessondate{08.09.2025}\\
	
	
	\section{Nature and the environment}
	\section{Globalization}
	
	
	\newpage
	
	
	% ==================== Small demo: show boxes sequentiell (orange -> red -> schwarz) ====================
	\section{Farb- und Box-Demonstration}

	
	\lessondate{24.08.2025}\\
	Hier ist ein bewusst lineares Beispiel: die Kästchen folgen nacheinander von Orange zu Rot zu Schwarz, mit genügend Abstand, damit sie nicht ineinander greifen.
	
	\begin{theorem}{Lineares Beispiel}
		Dies ist ein Demonstrations-Theorem. Die Rahmenfarbe ist orange, der Hintergrund des Kastens ist ein sehr heller Orangeton, so dass Inhalt und Rahmen klar getrennt bleiben.
	\end{theorem}
	
	\begin{example}{Anwendungsbeispiel}
		Dieses Beispiel folgt direkt dem Theorem. Die Rahmenfarbe ist rot; auch hier ist der Innenbereich nur leicht getönt, um Lesbarkeit und Druckverträglichkeit zu erhalten.
	\end{example}
	
	\begin{task}{Aufgabe — Abschluss}
		Dies ist die Aufgabenbox. Sie nutzt die fast schwarze Akzentfarbe. Aufgrund der erhöhten Kontraststärke ist der Innenraum nur sehr dezent getönt, damit Texte weiterhin im Vordergrund stehen.
	\end{task}
	
	\begin{solution}{Beispiel-Lösung}
		Kurze Musterlösung, die der Aufgabenbox folgt.
	\end{solution}
	
	% ==================== Rest des Dokuments (Beispielhafte Reihenfolge beibehalten) ====================
	\clearpage
	\section{Introduction to the Course}
	\lessondate{17.08.2025}\\
	Diese Version des Skripts priorisiert klare Reihenfolgen und visuelle Hierarchie: von warm (Orange) über kritisch (Rot) zu definitorisch/präskriptiv (Schwarz).
	
	% (der restliche Inhalt aus dem ursprünglichen Template kann hier wieder eingefügt werden)
	
\end{document}
