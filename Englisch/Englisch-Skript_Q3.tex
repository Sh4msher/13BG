% ============== Adjusted LaTeX Template (XeLaTeX / LuaLaTeX) ==============
\documentclass[11pt,a4paper,oneside]{article}

% -------------------- Engine & Fonts --------------------
\usepackage{fontspec}
\setmainfont{Libertinus Serif}
\setsansfont{Libertinus Sans}
\usepackage{unicode-math}
\setmathfont{Libertinus Math}

% -------------------- Packages --------------------
\usepackage{paralist}
\usepackage{tasks}
\usepackage{acro}
\usepackage{microtype}
\usepackage{geometry}
\usepackage{titlesec}
\usepackage{fancyhdr}
\usepackage{xcolor}
\usepackage{pagecolor}
\usepackage{tikz}
\usetikzlibrary{shadows,calc}
\usepackage[most]{tcolorbox}
\tcbuselibrary{skins,breakable,theorems}
\usepackage{enumitem}
\usepackage{caption}
\usepackage{everypage}
\usepackage{amsmath}
\usepackage{graphicx}

% -------------------- Layout --------------------
\geometry{
	left=28mm, right=28mm, top=28mm, bottom=28mm,
	marginparwidth=36mm, marginparsep=6mm
}

% -------------------- Colors (adjusted) --------------------
% Page background: slightly lighter yellow (helleres gelb)
\definecolor{PageBG}{RGB}{255,253,230} % very pale warm yellow
\definecolor{dings}{RGB}{255,255,240}
\definecolor{TextBlack}{RGB}{34,34,34}

% Accent palette (flow: orange -> red -> black)
\definecolor{AccentOrange}{RGB}{255,140,0}   % strong orange
\definecolor{AccentRed}{RGB}{220,20,60}      % crimson-ish red
\definecolor{AccentDark}{RGB}{20,20,20}      % near-black for final boxes

% Subtle box backgrounds (tints derived from accents)
\colorlet{BoxOrange}{AccentOrange!8!white}
\colorlet{BoxRed}{AccentRed!6!white}
\colorlet{BoxDark}{AccentDark!6!white}
\definecolor{MarginalGray}{RGB}{112,128,144}


\definecolor{TextCream}{RGB}{246,241,234}       % creme / sehr hell

\definecolor{AccentBlue}{RGB}{76,104,150}       % akademisches Blau (unverändert)
\definecolor{AccentViolet}{RGB}{105,77,145}     % seriöses Violett (unverändert)

% Hint: mehr ins violett/pinke ziehen (Hauch von Fuchsia)
\definecolor{AccentMagenta}{RGB}{140,60,120}       % violett-pink (sanfter Übergang)

% Magenta: rötlicher, dichter, aber nicht grell
\definecolor{AccentHint}{RGB}{170,40,60}     % rötliches Magenta (mehr Rotanteil)

% Elegantes Rot: orientiert sich am bisherigen "AccentHint" (tiefes Weinrot)
\definecolor{AccentRed}{RGB}{120,10,25}         % elegantes Weinrot / tiefes Rot

\definecolor{BoxBackground}{RGB}{15,9,27}       % Box-Hintergrund
\definecolor{MarginalGray}{RGB}{150,150,160}    % Rand-Datum




	
\pagecolor{dings}
\color{TextBlack}

% -------------------- Header / Footer --------------------
\pagestyle{fancy}
\fancyhf{}
\renewcommand{\headrulewidth}{0pt}
\setlength{\headheight}{14pt}
\fancyfoot[C]{%
	\vspace{6pt}%
	{\scriptsize\scshape\color{MarginalGray}Ferdinand-Braun Schule \quad • \quad Advanced English Course (Q3) \quad • \quad \thepage}%
	\begin{tikzpicture}[remember picture,overlay]
		\draw[line width=0.9pt,color=AccentOrange!70!black] ($(current page.south west)+(28mm,22mm)$) -- ($(current page.south east)+(-28mm,22mm)$);
	\end{tikzpicture}%
}

% -------------------- Title Styles (follow color flow) --------------------
\titleformat{\section}
{\normalfont\large\bfseries\color{AccentOrange!95!black}}
{\thesection}{1em}{}
\titleformat{\subsection}{\normalfont\normalsize\bfseries\color{AccentRed!85!black}}{\thesubsection}{0.8em}{}
\titleformat{\subsubsection}{\normalfont\normalsize\bfseries\color{AccentDark!85!black}}{\thesubsubsection}{0.6em}{}

% ========================= TCBOX BASE STYLE (sauber, nicht ineinander verwoben) =========================
\tcbset{
	mybase/.style={
		enhanced,
		breakable,
		boxrule=0.8pt,
		colframe=black!20,
		colback=white,
		colupper=TextBlack,
		arc=3mm,
		boxsep=6pt,
		left=14pt,right=14pt,top=12pt,bottom=12pt,
		before skip=12pt, after skip=12pt, % mehr Abstand, verhindert "ineinander verwoben"
		boxed title style={
			arc=4mm,
			boxrule=0pt,
			interior style={left color=white,right color=white},
			attach boxed title to top left={yshift=-\tcboxedtitleheight/2,xshift=8mm}
		},
		fonttitle=\sffamily\bfseries\small,
		title after break=\vspace{6pt}
	}
}

% -------------------- BOX TYPES (klarer, sequentieller Farbübergang) --------------------
\newtcolorbox[auto counter,number within=section]{theorem}[2][]{%
	mybase,
	colframe = AccentOrange!85!black,
	colback = BoxOrange,
	colbacktitle = AccentOrange!85!black,
	coltitle = PageBG,
	title = {Theorem~\thetcbcounter: #2},
	#1
}

\newtcolorbox[auto counter,number within=section]{example}[2][]{%
	mybase,
	colframe = AccentRed!85!black,
	colback = BoxRed,
	colbacktitle = AccentRed!85!black,
	coltitle = PageBG,
	title = {Example~\thetcbcounter: #2},
	#1
}

\newtcolorbox[auto counter,number within=section]{task}[2][]{%
	mybase,
	colframe = AccentDark!80!black,
	colback = BoxDark,
	colbacktitle = AccentDark!90!black,
	coltitle = PageBG,
	title = {Task~\thetcbcounter: #2},
	#1
}

\newtcolorbox[use counter from=task]{solution}[2][]{%
	mybase,
	colframe = AccentDark!80!black,
	colback = BoxDark,
	colbacktitle = AccentDark!90!black,
	coltitle = PageBG,
	title = {Solution~\thetcbcounter: #2},
	#1
}

% -------------------- Note Box (neutral, but follows dark accent) --------------------
\newtcolorbox{infobox}[1][]{%
	mybase,
	colframe = AccentDark!60!black,
	colback = BoxDark,
	colbacktitle = AccentDark!80!black,
	coltitle = PageBG,
	title = {Note},
	#1
}

\newcommand{\lessondate}[1]{
	\noindent\hfill\textcolor{gray}{\textsc{#1}} \\
	\vspace{0.5cm}
}



	
% ==================== Feines Titelblatt ====================
\newcommand{\MakeArtTitle}[4]{%
	\begin{titlepage}
		\vspace*{18mm}
		\begin{center}
			\vspace{12mm}
			{\huge\color{AccentOrange!90!black} #1 \par}
			\vspace{6mm}
			{\Large\itshape\color{AccentRed!70!black} #2 \par}
			\vspace{10mm}
			{\Large\scshape\color{AccentDark} #3 \par}
			\vspace{6mm}
			{\small\color{MarginalGray} #4 \par}
			\vspace{5mm}
			{\small\color{MarginalGray} \today \par}
		\end{center}
		\vspace{7.5cm}
		\centering
		\includegraphics[width=0.75\textwidth]{image.png} % Logo einfügen (Pfad anpassen)
	\end{titlepage}
}


% -------------------- Start Document --------------------
\begin{document}

	

	
		
	% Titelblatt
	\MakeArtTitle{
		Grundkurs Englisch Q3 Hessen}
	{Skript}
	{Shamsher Singh Kalsi}
	{Berufliches Gymnasium — Ferdinand-Braun Schule \\ Kursleiter: Herr Klaus Heinz}
	
	\tableofcontents
	\bigskip
	\clearpage
	
	
	
	\section{Einleitung}
	Für dieses Halbjahr wird das Buch The Circle von Dave Eggers behandelt. Nach den Herbstferien wird das Buch bearbeitet.
	
	\section{Human dilemmas in fiction and real life}
	
	
	
	
	\section{Modelling the future}
	\section{Gender issues}
	
	\lessondate{25.05.2025}\\
	
	Workbook Page 32. 
	
	\subsection{Diversity in the Workplace}
	\begin{itemize}
		\item Diversity means that a company is made up of a wide range of individuals - gender, race, age, educational background.
		\item Issues; Takes a lot of work to take care of the perspective of the others. maybe people in a team from the same background work more efficient 
		\item divers teams can be more creative and therefore more efficient to solve problems 
		\item downsides: increase the risk of disagreemends and delay projects
		\item cultural concepts: when people are not open enough to accept others opinions
		\item can be successful if everyone shows commitment and is accepting one another
		\item 
	\end{itemize}
	
	\subsubsection{Cartoon Analysis}
	\lessondate{01.09.2025}\\
	The present cartoon is made by Loren Fishman and is about discrimination and the lack of diversity in workplaces. 
	
	You can see a generic workplace with several men in the center of the cartoon. There are three pairs of men talking to each other -  one on the middle left, one on the middle right and one in the left front. Three men isolated and work by them self. All men look similiar with a black suit, a white shirt, a black tie and a slightly colored skin with the same haircut. All do share the same blank facial expression. 
	On the bottom of the cartoon is a quote "Oh, you will love it here. Nobody treats you any differently just because of your age, race or gender". 
	The first association is ofcourse the paradox that on one hand the cartoon is pointing out tolerance in the company while on the other hand everybody is the same. 
	
	
	\vspace{20mm}
	The cartoon is striking because every person looks completely interchangeable, as if individuality has been erased. Each character seems to be a product of the capitalist machine, designed not to think but to execute. Diversity, difference, and true personality have no place in such a world, because the system functions best when people can be treated identically, without responsibility or deeper recognition. The blank facial expressions highlight this dehumanization: workers are reduced to machines, stripped of identity. The cartoon cleverly exposes how capitalism prefers uniformity over individuality, efficiency over humanity, and empty words of equality over genuine diversity.
	
	
	\section{Nature and the environment}
	\section{Globalization}
	
	
	\newpage
	
	
	% ==================== Small demo: show boxes sequentiell (orange -> red -> schwarz) ====================
	\section{Farb- und Box-Demonstration}

	
	\lessondate{24.08.2025}\\
	Hier ist ein bewusst lineares Beispiel: die Kästchen folgen nacheinander von Orange zu Rot zu Schwarz, mit genügend Abstand, damit sie nicht ineinander greifen.
	
	\begin{theorem}{Lineares Beispiel}
		Dies ist ein Demonstrations-Theorem. Die Rahmenfarbe ist orange, der Hintergrund des Kastens ist ein sehr heller Orangeton, so dass Inhalt und Rahmen klar getrennt bleiben.
	\end{theorem}
	
	\begin{example}{Anwendungsbeispiel}
		Dieses Beispiel folgt direkt dem Theorem. Die Rahmenfarbe ist rot; auch hier ist der Innenbereich nur leicht getönt, um Lesbarkeit und Druckverträglichkeit zu erhalten.
	\end{example}
	
	\begin{task}{Aufgabe — Abschluss}
		Dies ist die Aufgabenbox. Sie nutzt die fast schwarze Akzentfarbe. Aufgrund der erhöhten Kontraststärke ist der Innenraum nur sehr dezent getönt, damit Texte weiterhin im Vordergrund stehen.
	\end{task}
	
	\begin{solution}{Beispiel-Lösung}
		Kurze Musterlösung, die der Aufgabenbox folgt.
	\end{solution}
	
	% ==================== Rest des Dokuments (Beispielhafte Reihenfolge beibehalten) ====================
	\clearpage
	\section{Introduction to the Course}
	\lessondate{17.08.2025}\\
	Diese Version des Skripts priorisiert klare Reihenfolgen und visuelle Hierarchie: von warm (Orange) über kritisch (Rot) zu definitorisch/präskriptiv (Schwarz).
	
	% (der restliche Inhalt aus dem ursprünglichen Template kann hier wieder eingefügt werden)
	
\end{document}
