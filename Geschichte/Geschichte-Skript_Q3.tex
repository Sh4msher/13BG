% ===================== Geschichts-Skript (History theme, typewriter accents) =====================
\documentclass[11pt,a4paper,oneside]{article}

% -------------------- Engine & Fonts --------------------
% WICHTIG: Dieser Block ist für XeLaTeX / LuaLaTeX gedacht.
% Wenn Sie pdfLaTeX verwenden, müssen Sie diesen Teil auskommentieren
% und alternative Pakete wie \usepackage[utf8]{inputenc} und \usepackage[T1]{fontenc} verwenden.
\usepackage{fontspec} % XeLaTeX / LuaLaTeX
\setmainfont{Libertinus Serif}
\setsansfont{Libertinus Sans}
\newfontfamily\typewriter{Courier New}
%[
%Path = ./, % Ggf. Pfad zu den Fonts anpassen
%UprightFont = *-Regular,
%ItalicFont = *-Italic
%]
\usepackage{unicode-math}
\setmathfont{Libertinus Math}

% -------------------- Pakete --------------------
%\usepackage[utf8]{inputenc}
%\usepackage{amsmath} % Mathematik-Pakete
%\usepackage{amsfonts}
%\usepackage{mathptmx} %times new roman
%\usepackage[T1]{fontenc} %vollen Zeichensatz ]
%\usepackage{amssymb}
%\usepackage{tcolorbox}
\usepackage[ngerman]{babel} 
\usepackage{graphicx}
\usepackage{microtype} %besserer Randausgleich
\usepackage{footnote}
\usepackage{blindtext}
\usepackage{etoolbox}
%\usepackage{makeidx}
%\usepackage{dsfont}
\usepackage{lettrine}%\usepackage{geometry}
%\usepackage{xcolor} % Für verschiedene Farben
\newcommand{\ricardo}[1]{\colorbox{ForestGreen}{\color{white}   \textsf{\textbf{Ricardo}}} \textcolor{ForestGreen}{#1}}
\usepackage[pdfborderstyle={/S/U/W 1}]{hyperref} % Für interaktive Refernzierung im PDF
\usepackage{csquotes}
\usepackage{acro}
\usepackage{hyperref} % Für interaktive Refernzierung im PDF
\usepackage[onehalfspacing]{setspace}%Zeilenabstand 1.5
% \usepackage{picins} % Das Umfließen einer Grafik im Text kann mit dem Paket PicIns erreicht werden.
%\usepackage{fontspec} 
\usepackage{mathtools}
%\usepackage[utf8]{inputenc}
\usepackage[ngerman]{babel}
\usepackage[top=3cm, bottom=3cm, left=2.5cm, right=2.5cm]{geometry}
\usepackage{bibgerm}
\usepackage{tabularx}
\usepackage{adjustbox}
\usepackage{cite}
\usepackage{blindtext}
\usepackage{epsfig}
\usepackage{longtable}
%\usepackage{showframe}
\usepackage{dcolumn}%benötigt für stargaze
\usepackage{here}%lädt das Paket zum Erzwinge n der Grafikposition
\usepackage{floatflt}%Bilder im Fließtext
%\usepackage{fontspec}
%\usepackage{fontenc}
\usepackage{dsfont}
%\setsansfont[Ligatures=TeX]{Arial}
%\renewcommand{\familydefault}{\sfdefault}
%\usepackage{times}
\usepackage{graphicx}
\usepackage{epstopdf}
\usepackage{physics}
\usepackage[siunitx]{circuitikz} %[symbols]
\usepackage{xcolor}
\usepackage{listings}

\usepackage{lipsum}

\usepackage[utf8]{inputenc}
\usepackage[T1]{fontenc}
\usepackage[ngerman]{babel}
\usepackage{amsmath}
\usepackage{tocloft}
\renewcommand{\cftsecleader}{\cftdotfill{\cftdotsep}} % Punkte zwischen Abschnittsnummern und -titeln



\usepackage{acro}
\usepackage[ngerman]{babel}
\usepackage{microtype}
\usepackage{geometry}
\usepackage{titlesec}
\usepackage{fancyhdr}
\usepackage{xcolor}
\usepackage{pagecolor}
\usepackage{tikz}
\usetikzlibrary{shadows,calc,decorations.pathmorphing}
\usepackage[most]{tcolorbox}
\tcbuselibrary{skins,breakable,theorems}
\usepackage{enumitem}
\usepackage{caption}
\usepackage{everypage}
\usepackage{graphicx}
\usepackage{ifxetex} % Nützlich für bedingte Kompilierung

% -------------------- Layout --------------------
\geometry{
	left=28mm, right=28mm, top=28mm, bottom=28mm,
	marginparwidth=36mm, marginparsep=6mm
}

% -------------------- Farben (historisch / sepia) --------------------
\definecolor{Parchment}{RGB}{245,238,224}     % helles Pergament
\definecolor{Ink}{RGB}{40,30,22}              % dunkle Tinte
\definecolor{Sepia}{RGB}{112,85,60}           % Akzent Sepia
\definecolor{DarkSepia}{RGB}{90,60,40}        % dunklerer Akzent
\definecolor{BoxBG}{RGB}{250,245,235}         % Box Hintergrund (leicht heller)
\definecolor{StampRed}{RGB}{110,20,20}        % Stempelfarbe
\definecolor{MarginalGray}{RGB}{120,110,100}

% wähle Hintergrund: Pergament-Look
\pagecolor{Parchment}
\color{Ink}

% -------------------- Kopf / Fuß --------------------
\pagestyle{fancy}
\fancyhf{}
\renewcommand{\headrulewidth}{0pt}
\setlength{\headheight}{14pt}
\fancyfoot[C]{%
	\vspace{6pt}%
	{\scriptsize\scshape Ferdinand-Braun Schule \quad • \quad Grundkurs Geschichte Q3 \quad • \quad \thepage}%
	\begin{tikzpicture}[remember picture,overlay]
		\draw[line width=0.9pt,color=DarkSepia] ($(current page.south west)+(28mm,22mm)$) -- ($(current page.south east)+(-28mm,22mm)$);
	\end{tikzpicture}%
}

% -------------------- Titel-Styles --------------------
\titleformat{\section}
{\normalfont\large\bfseries\color{DarkSepia}\typewriter}
{\thesection}
{1em}
{}
\titleformat{\subsection}{\normalfont\normalsize\bfseries\color{Sepia}\typewriter}{\thesubsection}{0.8em}{}
\titleformat{\subsubsection}{\normalfont\normalsize\bfseries\color{DarkSepia}}{\thesubsubsection}{0.6em}{}

% ========================= TCBOX BASIS-STIL (History) =========================
\tcbset{
	histbase/.style={
		enhanced,
		breakable,
		boxrule=0.6pt,
		colframe=DarkSepia,
		colback=BoxBG,
		colupper=Ink,             % Textfarbe innerhalb der Box
		arc=1mm,
		boxsep=6pt,
		left=12pt,right=12pt,top=12pt,bottom=12pt,
		before skip=8pt, after skip=8pt,
		attach boxed title to top left={yshift=-0.25mm-\tcboxedtitleheight/2, xshift=10mm},
		boxed title style={
			arc=1mm,
			left=6pt,right=6pt,top=4pt,bottom=4pt,
			boxrule=0pt
		},
		fonttitle=\typewriter\small,
		title after break=\vspace{4pt}
	}
}

% -------------------- Historische Boxen --------------------
% Hinweis / Kontext
\newtcolorbox{histnote}[1][]{%
	histbase,
	colframe = DarkSepia,
	colbacktitle = Sepia!20,
	coltitle = Ink,
	title = {Hinweis},
	#1
}

% Primärquelle (typewriter inside)
\newtcolorbox{primarysource}[2][]{%
	histbase,
	colframe = DarkSepia,
	colbacktitle = DarkSepia!10,
	coltitle = Ink,
	title = {Quelle: #2},
	fontupper=\typewriter\small,
	#1
}

% Timeline-Box (Datum prominent links)
\newtcolorbox{timeline}[2][]{%
	histbase,
	colframe = Sepia!60!black,
	colbacktitle = Sepia!15,
	coltitle = Ink,
	title = {#2},
	left=18mm,
	%underlay={
	%	\begin{tikzpicture}[overlay]
	%		% This part had a small bug, corrected to shade a vertical bar
	%		\fill[Sepia!30] (frame.south west) rectangle ($(frame.north west)+(12mm,0)$);
	%	\end{tikzpicture}
	%},
	#1
}


% Aufgabe / Lösung (farbliche Kennzeichnung)
\newtcolorbox[auto counter,number within=section]{histaufgabe}[2][]{%
	histbase,
	colframe = Sepia!70!black,
	colbacktitle = Sepia!25,
	coltitle = Ink,
	title = {Aufgabe~\thetcbcounter: #2},
	#1
}
\newtcolorbox[use counter from=histaufgabe]{histloesung}[2][]{%
	histbase,
	colframe = DarkSepia!75,
	colbacktitle = DarkSepia!20,
	coltitle = Ink,
	title = {Lösung~\thetcbcounter: #2},
	fontupper=\itshape,
	#1
}

% Quellen-Zitat (kleiner, in typewriter, mit Rand)
\newtcolorbox{quoteBox}[1][]{%
	colframe = DarkSepia!60,
	colback = BoxBG,
	colupper = Ink,
	boxrule=0.5pt,
	arc=1.5mm,
	left=10pt,right=10pt,top=8pt,bottom=8pt,
	fontupper=\typewriter\small,
	before skip=6pt, after skip=6pt,
	#1
}

% ------------------- Datum in rechter Margin (jede Seite) -------------------
%\newcommand{\lessondate}[1]{%
%	\AddEverypageHook{%
%		\begin{tikzpicture}[remember picture,overlay]
%			\node[anchor=north east,inner sep=0pt] at ($(current page.north east)+(-18mm,-14mm)$) {%
%				\parbox{36mm}{\raggedleft\small\sffamily\color{MarginalGray}#1}%
%			};
%		\end{tikzpicture}%
%	}%
%}

% Makro für das Datum am rechten Rand
\newcommand{\lessondate}[1]{
	\noindent\hfill\textcolor{gray}{\textsc{#1}} \\
	\vspace{0.5cm}
}


% ==================== Feines Titelblatt (History edition) ====================
\newcommand{\MakeArtTitle}[6][]{% KORRIGIERT: von [5] auf [6]
	% #1 opt: small subtitle, #2 Title, #3 Subtitle, #4 Author, #5 Course/Info, #6 Logo
	\begin{titlepage}
		\centering
		\vspace*{24mm}
		% large title block (typewriter headline)
		{\huge\typewriter\color{DarkSepia} #2 \par}
		\vspace{6mm}
		{\Large\itshape\color{Sepia} #3 \par}
		\vspace{12mm}
		{\Large\scshape\color{Ink} #4 \par}
		\vspace{4mm}
		{\small\color{MarginalGray} #5 \par}
		\vspace{12mm}
		% optional subtitle sticker
		\if\relax\detokenize{#1}\relax\else
		\vspace{6mm}
		\begin{tikzpicture}
			\node[draw=DarkSepia,fill=Sepia!10,rounded corners=2pt,inner sep=6pt] {\typewriter\small #1};
		\end{tikzpicture}
		\vspace{6mm}
		\fi
		% stamp / seal
		\vspace{18mm}
		\begin{tikzpicture}
			\node[inner sep=0pt] (logo) at (0,0) {};
			\draw[line width=1.2pt,color=StampRed!75,rotate=10] (0,0) circle (1.5cm);
			\node[rotate=10,font=\typewriter\small\color{StampRed!75}] at (0,0) {Archivexemplar};
			\draw[line width=0.6pt,color=StampRed!45,rotate=10] (0,0) circle (1.2cm);
		\end{tikzpicture}
		\vfill
		% optional logo graphic (falls vorhanden)
		\ifx\relax#6\relax\else
		\centering\includegraphics[width=0.4\textwidth]{#6}
		\fi
		\vspace{3cm}
		\centering
		\includegraphics[width=0.75\textwidth]{image.png} % Logo einfügen (Pfad anpassen)
	\end{titlepage}
}

% ==================== Dokumentbeginn ====================
\begin{document}
	
	% Titelblatt aufrufen: KORRIGIERT - Leeres Argument {} hinzugefügt
	\MakeArtTitle[Sonderausgabe]{Grundkurs Geschichte Q3 Hessen}{Ost-West-Konflikt, postkoloniale Welt und Globalisierung Skript}{Shamsher Singh Kalsi}{Berufliches Gymnasium — Ferdinand-Braun Schule \\ Kursleiterin: Frau Dr. Braun}{}
	\tableofcontents
	\clearpage
	
	% Datum rechts
	\lessondate{18.08.2025}
	
	\section{Einleitung}
	Dieses Skript sammelt Zusammenfassungen, Quellen und Arbeitsaufträge für den Geschichtsgrundkurs.
	Die Gestaltung lehnt sich an Archivdokumente an: \emph{typewriter} für Primärtexte / Überschriften, sepia-Akzente und Pergament-Hintergrund.
	
	
	\subsection{Organisatorisches}
	\lessondate{19.09.2025}\\
	Es wurde ein neuer Moodle Kurs für die Q3 von der Kursleiterin angelegt. Für die Bearbeitung der Aufgaben verlangt die Kursleiterin genaues lesen der Aufgabenstellung und sich dementsprechend vorzubereiten im Sinne mit einer Tabelle, Mind map und Concept map. Dies soll helfen Zusammenhänge und Hintergründe effektiver darstellen zu können.  Es soll keine Aufzählung von Fakten sein, sondern eine Art Erörterung mit Fachbegriffen und Quellen belegen.  Quellenbezüge sollen mit einem Kontext integriert werden.
	
	\subsection*{Schriftliches und Mündliches Abitur}
	Bestehend aus drei Anfordungsbereiche
	\begin{enumerate}
		\item Reproduktion 
		\item Zusammenhänge herstellen
		\item Relfexion und Verallgemeinerung 
	\end{enumerate}
	
	\subsection{Beispiel Übung}
	
	\begin{histaufgabe}{Textüberarbeitung}
		Zur Aufgabe (Stellen Sie die Innenpolitik Bismarcks dar) wurde der folgende Text verfasst. Überarbeiten Sie ihn, indem Sie problematische Textstellen markieren, fehlende Aspekte ergänzen und Verbesserungen notieren.
	\end{histaufgabe}
	
	
	\newpage
	
	
	\section{Wiederholung aus der Q2}
	
	\lessondate{26.08.2025}\\

		
	\begin{histaufgabe}{Reproduzieren der Inhalte von einem Video}
		\begin{enumerate}
			\item Erklären Sie diese drei Begriffe auf Basis des Videos.
			\item Was wollte Lenin in Russland erreichen/ wie wollte er vorgehen und warum?
		\end{enumerate}
	\end{histaufgabe}
	
	\begin{histloesung}{Lösungsskizze}{Aufgabe 1}
		\subsection*{Kommunismus}
		Der Kommunismus ist eine politische und gesellschaftliche Idealvorstellung, in der alle Produktionsmittel (z.\,B. Fabriken, Rohstoffe, Ackerland) allen Menschen gemeinsam gehören. Privateigentum an Produktionsmitteln wird abgeschafft, das Produzierte soll gerecht verteilt werden. Nach der Lehre von Karl Marx und Friedrich Engels soll jeder nach seinen Fähigkeiten arbeiten und nach seinen Bedürfnissen nehmen können. Ziel ist eine klassenlose Gesellschaft.\\
		
		\subsection*{Sozialismus}
		Der Sozialismus ist nach Marx die notwendige Übergangsphase vom Kapitalismus zum Kommunismus. In dieser Phase werden die Produktionsmittel verstaatlicht und der Staat plant zentral, was produziert wird. Es gibt keine Kapitalisten mehr, die allein von Besitz leben, aber noch eine staatliche Ordnung, die Konflikte regelt. Der Sozialismus soll die soziale und materielle Lage der Arbeiter verbessern und so den Weg in den Kommunismus öffnen.\\
		
		\subsection*{Bolschewismus}
		Der Bolschewismus ist die von Lenin entwickelte und radikalisierte Form des Marxismus in Russland. Er bezeichnet sowohl die politische Lehre als auch die Organisationsform einer Kaderpartei von Berufsrevolutionären. Ziel war es, die kapitalistische Phase zu überspringen und durch eine Revolution direkt den Sozialismus zu errichten. Die Bolschewiki setzten dabei auf Gewalt, Terror und eine straffe Parteidiktatur, um die Gesellschaft nach kommunistischen Ideen umzugestalten.\\
	\end{histloesung}
	
	\newpage
	
	
	\begin{histloesung}{Aufgabe 2}
		Lenin strebte eine radikale Transformation Russlands an, um eine sozialistische Gesellschaft zu errichten, die letztlich den Weg zum Kommunismus ebnen sollte. Sein primäres Ziel war es, die veraltete Ordnung des Zarenreiches, welche er als rückständig, feudalistisch und repressiv betrachtete, zu stürzen und durch die Herrschaft der Arbeiter und Bauern zu ersetzen.
		
		Obwohl seine Ideologie auf den Lehren von Karl Marx basierte, wich er in einem fundamentalen Aspekt ab: Marx hatte argumentiert, dass der Sozialismus nur aus einem hoch entwickelten Kapitalismus hervorgehen könne. Im Gegensatz dazu war Russland zu Beginn des 20. Jahrhunderts überwiegend ein Agrarstaat, in dem eine kleine Elite dominierte und Industrie sowie Proletariat nur schwach ausgeprägt waren. Lenin stand somit vor einem Dilemma: Ein Abwarten des westlichen Entwicklungsmodells, also Kapitalismus hin zu Sozialismus und Kommunismus, hätte die fortgesetzte Unterdrückung und Abhängigkeit Russlands bedeutet.
		
		Um dieses Problem zu lösen, entschied sich Lenin, diesen Entwicklungsschritt zu überspringen. Er war überzeugt, dass eine straff organisierte revolutionäre Avantgarde-Partei den historischen Entwicklungsprozess beschleunigen könnte. Diese Partei, die Bolschewiki, bestand aus Berufsrevolutionären, die sich der Revolution vollkommen verschrieben hatten und die Aufgabe hatten, die Massen der Arbeiter und Bauern anzuführen.
		
		Sein Vorgehen beruhte auf zwei zentralen Säulen: Erstens, der gewaltsame und unmittelbare Umsturz des Zarenreiches sowie seiner politischen Institutionen. Zweitens, der Aufbau einer zentralistisch geführten Diktatur des Proletariats, die er als notwendige Übergangsphase ansah. Lenin war der Überzeugung, dass der Sozialismus nicht durch bloße Überzeugung oder Reformen etabliert werden könne, sondern durch Zwang und Terror durchgesetzt werden müsse, um den Widerstand der alten Eliten und politischen Gegner zu brechen.
		
		Die Triebfeder für diesen Weg war Lenins Ungeduld. Er wollte nicht auf die "naturgesetzlich" erwartete Entwicklung warten, sondern diese sofort in Russland erzwingen, obwohl das Land ökonomisch noch nicht reif war. Er nutzte die internationale Lage – den Ersten Weltkrieg, die Krise der Monarchien und die Not der Bevölkerung – als historisches Zeitfenster. Er glaubte, dass durch das Ausnutzen dieser Gelegenheit Russland nicht nur seinen eigenen Rückstand aufholen, sondern auch zum Vorreiter einer globalen Revolution avancieren könnte.
	\end{histloesung}
	
	
	\newpage
	
	\lessondate{26.08.2025}\\
	
	\subsection*{Kommunismus}
	\begin{itemize}
		\item gesell. Idealvorstellung 
		\item klassenlose Gesellschaft 
		\item kein Privateigentum 
		\item Produktionsmittel gehören allen 
		\item Ziel des Marxismus: Historischer Materialismus $\rightarrow$ naturgesetzliche Entwicklung 
	\end{itemize}
	
	\subsection*{Sozialismus}
	\begin{itemize}
		\item Vorstufe des Kommunismus
	\end{itemize}
	
	\subsection*{Bolchewismus}
	\begin{itemize}
		\item Problem: In Russland 
	\end{itemize}
	
	\newpage
	
	\section{Stalin's Aufstieg}
	
	\begin{histaufgabe}{l}
		\begin{itemize}
			\item Erklären Sie den Aufstieg Stalins
			\item Charakterisieren Sie Stalins Herrschaft und deren Hintergründe 
		\end{itemize}
	\end{histaufgabe}
	
	\begin{histloesung}{Aufgabe 1}
			\begin{itemize}
			\item Nach Lenins Tod 1924 entbrannte innerhalb der Kommunistischen Partei ein innerparteilicher Machtkampf zwischen verschiedenen Fraktionen (Linke um Trotzki, frühere Zinovjew–Kamenew-Gruppe, und später Rechte um Bukharin).\footnote{Stephen Kotkin, \textit{Stalin: Paradoxes of Power, 1878–1928}; Cambridge Univ. Press, Kapitel zu 1924–1928.}
			\item Stalin nutzte seine seit April 1922 bekleidete Funktion als Generalsekretär gezielt zur Kontrolle der Personalpolitik: durch die Verwaltung von Listen, Postenbesetzungen und die allmähliche Besetzung des Parteiapparats mit loyalen Kadern (das frühe Nomenklatura-Prinzip).\footnote{Ausführlich zur Rolle des Generalsekretärs und der Personalpolitik siehe Cambridge Univ. Press und Quellen zur Nomenklatura.}
			\item Taktisch bildete Stalin zunächst Bündnisse innerhalb der Partei — zuerst mit Zinovjev und Kamenev, später mit Bukharin — um die Linke (vor allem Trotzki) schrittweise zu marginalisieren und aus den Schlüsselpositionen zu drängen.\footnote{Siehe Darstellung der Fraktionspolitik 1924–1927 in Kotkin; ebenso Cambridge University Press, Kapitel \"Stalin's Rise to Power, 1924–29\".}
			\item Nachdem die Linke innerparteilich ausgeschaltet war, wandte sich Stalin gegen seine früheren rechten Verbündeten, um grundlegende Richtungswechsel (Ende der NEP, forcierte Industrialisierung und Kollektivierung) durchzusetzen; bis 1928/1929 hatte er die wichtigsten innerparteilichen Gegner weitgehend beseitigt.\footnote{Analyse des Bruchs mit der Rechten und der Politikwende gegen Ende der 1920er: Cambridge Univ. Press; Sekundärliteratur von Kotkin.}
			\item Die zentralistische, bürokratische Struktur der bolschewistischen Partei und die Kontrolle über Personalentscheidungen machten es möglich, oppositionelle Strukturen sukzessive zu schwächen; daraus entwickelte sich in den 1930er Jahren eine Politik der Zwangsmaßnahmen und Säuberungen, die die Parteiführung konsolidierte.\footnote{Zum Zusammenhang von Parteizentralismus, Personalpolitik und den späteren Säuberungen vgl. Kotkin; New Yorker-Essay zu den Archivbefunden; Studien zur Nomenklatura.}
		\end{itemize}
	\end{histloesung}
	
	\newpage
	
	\lessondate{02.09.2025}\\
	
	\begin{histaufgabe}{Buch Seite 335 - Sowjetische Expansion in Europa 1940 - 1947}
		Erläutern Sie anhand des Darstellungtextes sowie M1 die Grundprinzipien der Außenpolitik der Sowjetunion. 
	\end{histaufgabe}
	
	
	\begin{histloesung}{Buch Seite 335}
		\subsection*{Machausdehnung in Osteuropa}
		\begin{itemize}
			\item Sowjetunion hatte aufgrund einer infrastruktruellen Schwäche nach dem Krieg eine geringe weltweite Machtpolitik als die USA 
			\item Es gab massives Missvertrauen und differenzierte Vorstellungen eines Idealen friedens 
			\item Russland wollte sich durch die Sicherung Bulgariens, Rumäniens und Polens eine art Gürtel bilden
			\item Die Alliierten nahmen jedoch mehr Druck war
			\item 
		\end{itemize}
	\end{histloesung}
	
	\begin{histloesung}{Buch Seite 383 und 384} 
		\subsection*{Kontroverse}
		\begin{itemize}
			\item Die USA und Russland vertraten unterschiedliche Vorstellungen von Weltfrieden: 
			\item Die USA stellte sich einen freien Weltmarkt und Demokratie vor. Anscheinend war die Sowjetunion zu expansitionisitisch orientiert. 
		\end{itemize}
		\subsection*{Ziele der Alliierten}
		\subsection*{Alliierte Kriegkonferenzen}
	\end{histloesung}
	
	
	\newpage
	
	\begin{histloesung}{Tafel}
		\subsection*{Ausgangssituation 1944/45: Schwächung durch 2. Weltkrieg}
		\begin{itemize}
			\item Sicherheitsbedürfnis
			\item Westmächte gehen aus Sicht der SU nicht auf ihre Bedürfnisse ein
			\item besetzte Gebiete werden annektiert (Teile Polens $\rightarrow$ Westverschiebung u. Polens Ostgrenze, Baltikum, Teile v. Tschechoslowakei + Rumäniens, Finnlands, Teile Ostpreußens)
			\item Schaffung eines RIngs von Satellitenstaaten (Polen, Ungarn, Tschechoslowkei, Rumänien, Bulgarien u.a.s)
			\item Gegenseitiges Misstrauen von West und Ost wird spürbar.
		\end{itemize}
	\end{histloesung}
	
	\lessondate{09.09.2025}\\
	
	\begin{histloesung}{Tafel}
		\begin{itemize}
			\item \textbf{2. Weltkrieg:} Anti -Hitler-Koalition - USA, GB, SU $\Rightarrow$ gemeinsamer Feind überwiegte Konflikte
			\item \textbf{1945:} (nach und nach) Zerfall der Anto-Hitler-Koalition 
			\begin{itemize}
				\item \textbf{USA:} Atlantik
				\begin{itemize}
					\item freier Handel 
					\item Selbstbestimmungsrecht der Völker 
					\item Demokratie
				\end{itemize}
				\item \textbf{SU}
				\begin{itemize}
					\item Sicherheitsbedürftnis
					\item Weltmacht aufbauen / sichern 
					\item Kommunismus
				\end{itemize}
				\item [$\Rightarrow$] Entwicklung wird verstärkt durch US-Besitz der Atombombe u. Wechsel zu wenig kompromissbereiten US-Präsidenten 
				\item [$\Rightarrow$] Wachsende beidseitiges Misstrauen
			\end{itemize} 
		\end{itemize}
	\end{histloesung}
	
	
	\newpage
	
	
	
	\begin{center}
		\begin{minipage}[t]{0.47\textwidth}
			\subsection*{Harry S. Truman}
			\begin{itemize}
				\item Die Vereinigten Staaten sehen sich in einer führenden Rolle bei der Sicherung des internationalen Friedens.
				\item Totalitäre Regierungsformen werden anderen Völkern ohne deren Zustimmung aufgezwungen und bedrohen die Freiheit.
				\item Staaten müssen sich entscheiden: entweder für eine freie, demokratische Lebensform oder für eine restriktive, kommunistische Lebensform.
				\item Die Vereinten Nationen können ihren Zweck nur erfüllen, wenn sie Staaten dabei unterstützen, dieselben Werte wie die USA – Freiheit und Demokratie – zu verwirklichen.
			\end{itemize}
		\end{minipage}\hfill
		\begin{minipage}[t]{0.47\textwidth}
			\subsection*{Andrei A. Schdanow}
			\begin{itemize}
				\item Die Welt ist in zwei Lager gespalten: das imperialistisch-kapitalistische unter Führung der USA und das antiimperialistisch-demokratische unter Führung der UdSSR.
				\item Die USA verfolgen eine Politik der Expansion, Unterdrückung und ökonomischen Ausbeutung anderer Staaten.
				\item Das sowjetische Lager steht für Frieden, Demokratie und nationale Unabhängigkeit.
				\item Alle progressiven und friedliebenden Kräfte müssen sich im Kampf gegen Imperialismus und für die internationale Solidarität zusammenschließen.
			\end{itemize}
		\end{minipage}
	\end{center}
	
	
	
	
	
	
	\section{Der kalte Krieg - stabile oder labile Weltordnung?}
	\section{Die Teilung Deutschlands - eine Nation, zwei Staaten}
	\section{Deutschland von der Teilung zur Einheit}
	\section{Weltpolitische Entwicklungen zwischen Bipolarität und Multipolarität}
	\section{Der Nahostkonflickt als weltpoltischer Krisenherd}
	\section{Umgang mit der nationalsozialistischen Vergangenheit - "Vergangenheitsbewältigung" ?}
	
	\newpage
	
	\section{Beispielboxen}
	
	\begin{histnote}
		Dies ist ein kontextgebender Hinweis – z. B. ein kurzer historiographischer Abriss oder ein wichtiger Hinweis für die Lehrkraft.
	\end{histnote}

	
	\begin{primarysource}{Friedrich Schlegel, 1798}
		„Dies ist ein Beispieltext, der im Schreibmaschinenstil dargestellt wird, um Primärquellen-Feeling zu erzeugen.
		Datum: 12. April 1798. Ort: Berlin.“
	\end{primarysource}
	
	\begin{timeline}{1789 — Französische Revolution}
		Stichpunkte zu Ablauf und wichtigen Daten:
		\begin{itemize}
			\item 5. Mai 1789: Einberufung der Generalstände.
			\item 14. Juli 1789: Sturm auf die Bastille.
			\item 1792: Sturz der Monarchie.
		\end{itemize}
	\end{timeline}
	
	\begin{histaufgabe}{Quellenanalyse}
		Lest die Quelle (S. 12) und analysiert Ansatzpunkte für die Entstehung von Nationalbewusstsein.
	\end{histaufgabe}
	
	\begin{histloesung}{Lösungsskizze}
		Mögliche Stichpunkte: Bildung von Identität durch gemeinsame Sprache, Revolution als kollektives Ereignis, ...
	\end{histloesung}
	
	\begin{quoteBox}
		„War die Revolution notwendig? — Diese Frage bleibt bis heute Gegenstand intensiver Debatten.“ — Lehrbuchzitat
	\end{quoteBox}
	
	Megalolz
	
\end{document}